% ex: ts=2 sw=2 sts=2 et filetype=tex
% SPDX-License-Identifier: CC-BY-SA-4.0

\begin{frame}
  \frametitle{Contenido}
  \tableofcontents
\end{frame}

\section{¿Qué son y qué hacen?}

\begin{frame}{¿Qué es un sistema operativo?}
  \begin{itemize}
    \item El \emph{sistema base} de una computadora \pausa
    \item El programa que \emph{siempre corre} \pausa
    \item Gestor de los \emph{recursos} del sistema \pausa
    \item Lo que define \emph{qué es compatible} y \emph{qué no} dentro de una
          determinada arquitectura \pausa
    \item El programa \emph{menos importante} de todos \pausa
        \begin{itemize}
          \item No realiza \emph{trabajo útil}, sino que permite que \emph{otros} lo hagan
        \end{itemize}
    \item …
  \end{itemize}
\end{frame}

\begin{frame}{…¿Qué \emph{no} es?}
  \begin{itemize}
    \item Los programas básicos para administrar archivos
    \item La forma en que el usuario \emph{lanza} programas
    \item El ambiente con que interactúa el usuario
        \begin{itemize}
            \item Entorno gráfico
            \item Línea de comando
            \item …
        \end{itemize}
    \end{itemize}
\end{frame}

\begin{frame}{¿Qué brinda al \emph{programador}?}
  \begin{itemize}
    \item Abstracción
        \begin{itemize}
          \item \emph{Embellecimiento} -Finkel
          \item \emph{Virtualización} -Arpaci-Dusseau
        \end{itemize}
    \item Gestión de recursos
    \item Aislamiento y protección
  \end{itemize}
\end{frame}

\begin{frame}{Abstracción}
  \begin{center}
    Proporciona \emph{abstracciones consistentes} y \emph{simplificaciones} a los
    procesos del usuario
  \end{center}
  \begin{itemize}
    \item Archivos y directorios
    \item \emph{Flujos de caracteres} (entrada/salida)
    \item Dispositivos, conexiones de red, contacto con el \emph{mundo exterior}
    \item El concepto mismo de \emph{proceso}
  \end{itemize}
\end{frame}

\begin{frame}{Gestión de recursos}
  \begin{center}
    Administra los \emph{recursos existentes} en la computadora, permitiendo la
    ejecución a los diversos procesos
  \end{center}
  \begin{itemize}
    \item Cómo comparten los diversos procesos los recursos existentes (y
          rivales)
    \item Políticas de asignación (y recuperación) justas
  \end{itemize}
\end{frame}

\begin{frame}{Aislamiento y protección}
  \begin{center}
      Protección de los datos, de los recursos, de los procesos
  \end{center}
  \begin{itemize}
    \item Entre procesos
    \item Entre usuarios
    \item Ante procesos \emph{mal comportados}
    \item Ante procesos maliciosos
  \end{itemize}
\end{frame}

\begin{frame}{Ahora sí… ¿Qué es un sistema operativo?}
  \begin{itemize}
    \item El principal programa que corre en una computadora de propósito
          general
    \item Provee una serie de abstracciones básicas a los programas
          \begin{itemize}
            \item Pueden haber diferentes sistemas operativos, definiendo distintas
                  \emph{interfaces}, sobre el mismo hardware
            \item Un mismo sistema operativo puede \emph{adecuarse a} distintas
                  arquitecturas de hardware
          \end{itemize}
    \item Ofrece una infraestructura de gestión, aislamiento y protección de
        recursos
    \item Permite la implementación de \emph{entornos operativos}
  \end{itemize}
\end{frame}
