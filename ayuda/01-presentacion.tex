% Created 2023-02-06 Mon 12:03
% Intended LaTeX compiler: pdflatex
\documentclass[presentation]{beamer}
\usepackage[utf8]{inputenc}
\usepackage[T1]{fontenc}
\usepackage{graphicx}
\usepackage{longtable}
\usepackage{wrapfig}
\usepackage{rotating}
\usepackage[normalem]{ulem}
\usepackage{amsmath}
\usepackage{amssymb}
\usepackage{capt-of}
\usepackage{hyperref}
\usepackage[spanish]{babel}
\usepackage{listings}
\pgfdeclareimage[height=0.7\textheight]{../img/pres/cintillo.png}{../img/pres/cintillo.png}\logo{\pgfuseimage{../img/pres/cintillo.png}}
\AtBeginSection[]{ \begin{frame}<beamer> \frametitle{Índice} \tableofcontents[currentsection] \end{frame} }
\definecolor{string}{rgb}{0,0.6,0} \definecolor{shadow}{rgb}{0.5,0.5,0.5} \definecolor{keyword}{rgb}{0.58,0,0.82} \definecolor{identifier}{rgb}{0,0,0.7}
\renewcommand{\ttdefault}{pcr}
\lstset{basicstyle=\ttfamily\scriptsize\bfseries, showstringspaces=false, keywordstyle=\color{keyword}, stringstyle=\color{string}, identifierstyle=\color{identifier}, commentstyle=\mdseries\textit, inputencoding=utf8, extendedchars=true, breaklines=true, breakatwhitespace=true, breakautoindent=true, numbers=left, numberstyle=\ttfamily\tiny\textit}
\newcommand{\rarrow}{$\rightarrow$\hskip 0.5em}
\usetheme{Warsaw}
\usecolortheme{lily}
\author{Gunnar Wolf}
\date{}
\title{Sistemas Operativos (FI-UNAM 0840): Presentación del curso}
\hypersetup{
 pdfauthor={Gunnar Wolf},
 pdftitle={Sistemas Operativos (FI-UNAM 0840): Presentación del curso},
 pdfkeywords={},
 pdfsubject={},
 pdfcreator={Emacs 28.2 (Org mode 9.5.5)}, 
 pdflang={Spanish}}
\begin{document}

\maketitle

\section{Punto de partida}
\label{sec:orgaabd153}

\begin{frame}[label={sec:org4a48afe}]{Mis coordenadas}
\begin{center}
Personales
\end{center}

\begin{description}
\item[{Nombre}] Gunnar Eyal Wolf Iszaevich
\item[{E-mail}] sistop@gwolf.org \(\leftarrow\) \uline{ese} correo, no gwolf@gwolf.org
\item[{Ubicación}] Instituto de Investigaciones Económicas UNAM
(Secretaría Técnica)
\item[{Teléfono}] 55-5623-0154 (IIEc-UNAM)
\end{description}

\begin{center}
Del curso
\end{center}

\begin{description}
\item[{Página Web}] \url{http://gwolf.sistop.org/}
\item[{Repositorio}] \small \url{https://github.com/unamfi/sistop-2023-1/}
\end{description}
\end{frame}

\begin{frame}[label={sec:org5fb3b3f}]{Horario, calendario}
\begin{itemize}
\item Martes y jueves
\item 17:30 a 19:30
\item Salón B-205
\begin{itemize}
\item En caso de ser necesario, videoconferencia en
\url{https://clase.sistop.unam.mx/sistop-2023-1}
\end{itemize}
\item 64 horas clase en total
\begin{itemize}
\item \emph{Perdemos} por feriados las clases del 15 de septiembre y del 1 de
noviembre \(\rightarrow\) \alert{60 horas} \emph{efectivas}.
\item Voy a ausentarme (por participacióne en un congreso) las clases
del 6 y 8 de septiembre (pero pienso cubrir esas ausencias\ldots{})
\end{itemize}
\item La evaluación principal será con trabajos prácticos \emph{en la medida de
lo posible}
\begin{itemize}
\item Posiblemente apoyándonos en un examen parcial (detalles más adelante)
\end{itemize}
\end{itemize}
\end{frame}

\section{Encuadre del curso}
\label{sec:orgea99ad2}

\begin{frame}[label={sec:org1f2323e}]{El curso dentro de la currícula}
\centering Plan 2016
\begin{center}
\includegraphics[width=0.5\textwidth]{../img/pres/mapa_curricular_2016.png}
\end{center}
\end{frame}

\begin{frame}[label={sec:org131db96}]{Seriación y materias relacionadas}
\begin{itemize}
\item Seriación obligatoria: \emph{Estructura y programación de computadoras}
\item Asumo familiaridad con otras materias:
\begin{itemize}
\item Fundamentos de programación
\item Estructuras de datos y algoritmos
\item Programación orientada a objetos
\end{itemize}
\end{itemize}
\end{frame}

\begin{frame}[label={sec:orgf8deb96}]{Lenguajes de programación}
\begin{itemize}
\item Familiaridad con algún lenguaje de programación de alto nivel
\begin{itemize}
\item Para seguir ejemplos (que serán principalmente en shell POSIX,
C, Ruby y Python)
\item Para hacer ejercicios en clase y examen (basta con pseudocódigo
semi-formal)
\item Para tareas (¡código legal/válido!)
\end{itemize}
\item Familiaridad básica con C
\begin{itemize}
\item Más para leer que para desarrollar
\end{itemize}
\end{itemize}
\end{frame}

\begin{frame}[label={sec:org045877a}]{Otros requisitos}
\begin{center}
Linux (GNU) / Unix
\end{center}
\begin{itemize}
\item \alert{Muy} conveniente tener acceso a un sistema basado en Linux, o algún
Unix
\item \alert{Muy} preferentemente, software libre
\item Tip: Si no lo tienes y no quieres hacer una instalación completa,
instálalo en una \emph{máquina virtual}
\begin{itemize}
\item \ldots{}Ya luego lo tomarás como entorno primario ;-)
\end{itemize}
\end{itemize}

\begin{center}
Lectura en inglés
\end{center}
\begin{itemize}
\item Buena parte del material de referencia es en inglés
\item El material de estudios de caso casi siempre es en inglés
\item Nivel de comprensión de lectura \alert{muy} recomendado
\end{itemize}
\end{frame}

\begin{frame}[label={sec:org181d05a}]{¿Por qué Linux/Unix?}
\begin{itemize}
\item Windows, MacOS son sistemas válidos y útiles
\item Los encontrarán con frecuencia en su vida diaria/futura
\item Pero para aprender\ldots{}
\end{itemize}
\pause
\centering \vskip 1em {\Large ¡Tarea!} \vskip 1em

Leer el texto \href{http://www.joelonsoftware.com/articles/Biculturalism.html}{\emph{Biculturalism}, de Joel Spolsky} (2003) para discutir en
clase.
\small

\url{http://www.joelonsoftware.com/articles/Biculturalism.html}
\end{frame}

\begin{frame}[label={sec:org6cf3a52}]{Programa de estudio}
\begin{center}
\begin{center}
\includegraphics[width=0.8\textwidth]{../img/pres/prog_estudio_2016.png}
\end{center}
\end{center}
\end{frame}
\section{Enfoque personal}
\label{sec:orgaac3cf5}
\begin{frame}[label={sec:org5a4a83b}]{¿Quién soy y por qué estoy aquí?}
\begin{itemize}
\item Formación autodidacta
\begin{itemize}
\item La necesidad es el mejor motor para aprender algo
\item ¿No conocemos algo que necesitamos? Lo aprendemos sobre la marcha
\item ¿Aprender algo para lo que no tengo uso? Aprendizaje destinado al
olvido/fracaso
\end{itemize}
\item Usuario, promotor y desarrollador de software libre
\begin{itemize}
\item Promotor y organizador de congresos de tinte académico y
construcción de comunidades en México desde el 2002
\item Desarrollador del proyecto Debian desde el 2003
\item Profesor de asignatura en la FI desde el semestre 2013-2
\end{itemize}
\end{itemize}
\end{frame}

\begin{frame}[label={sec:org5885d4a}]{¿Por qué me parece importante la materia?}
\begin{itemize}
\item No espero que se dediquen a \emph{escribir} sistemas operativos (aunque
definitivamente puede ocurrir)
\begin{itemize}
\item No es tan poco probable como creen\ldots{}
\end{itemize}
\item Pero en cualquier área de aplicación profesional \emph{requerimos conocer
su funcionamiento} para desempeñarnos mejor
\begin{itemize}
\item Citando el objetivo institucional de la materia, \emph{el alumno
obtendrá las bases para administrar un sistema operativo}, así
como \emph{diseñar y desarrollar software operativo}
\end{itemize}
\item Lo que veamos en esta materia tendrá aplicación prácticamente en
cualquier área de desempeño profesional
\end{itemize}
\end{frame}

\begin{frame}[label={sec:orgb4981fd}]{¿Qué espero que logremos?}
\begin{center}
\scriptsize
Adicionalmente al objetivo formal\ldots{}
\end{center}
\begin{itemize}
\item El alumno conocerá el desarrollo histórico de los sistemas
operativos, lo que le llevará a comprender la razón de ser y el
funcionamiento general de los diversos componentes de los sistemas
operativos actuales.
\item Aplicando el conocimiento obtenido sobre el funcionamiento general
de los sistemas operativos, el alumno podrá sacar mejor provecho de
la computadora.
\begin{itemize}
\item Al emplearla como usuario final, al administrarla y al programar
\end{itemize}
\item El alumno conocerá las principales herramientas que ofrecen los
sistemas operativos libres para el monitoreo y administración.
\end{itemize}
\end{frame}

\begin{frame}[label={sec:org0e1dba5}]{¿Qué espero que logremos?}
\begin{center}
\emph{Desmitificar} al uso del cómputo
\end{center}
\begin{itemize}
\item ¿Por qué se \emph{congela} la computadora?
\item ¿Qué es eso de \emph{expulsar con seguridad} una unidad?
\item ¿Por qué una computadora con más memoria \emph{parece ser} más rápida?
\item ¿Qué pasa si reinicias y lo vuelves a intentar?
\item \ldots{}
\end{itemize}
\end{frame}

\section{Estructura del curso}
\label{sec:org4a1a872}
\begin{frame}[label={sec:orgf8e6e54}]{Unidades}
\begin{enumerate}
\item Introducción a los sistemas operativos
\item Relación con el hardware: Estructuras y funciones básicas
\item Administración de memoria
\item Administración de procesos
\item Planificación de procesos
\item Sistemas de archivos
\end{enumerate}
\end{frame}

\begin{frame}[label={sec:orga8709a2}]{Temas transversales}
\begin{center}
El temario propuesto de la Facultad contempla los siguientes temas
como unidades independientes:
\end{center}
\begin{enumerate}
\item Sistemas de entrada/salida
\item Sistemas distribuidos
\item Seguridad y medidas de desempeño
\end{enumerate}
\begin{center}
Estos temas serán abordados de forma \emph{transversal}, esto es,
son temas que implican a los demás subsistemas y considero que no
pueden ser estudiados de forma aislada.
\end{center}
\end{frame}

\begin{frame}[label={sec:orgfb7f140}]{Adicionalmente\ldots{}}
\begin{itemize}
\item Presentaré algunos temas relevantes a la materia más cercanos a la
actualidad en el campo
\item Buscaremos presentar ejemplificando
\begin{itemize}
\item Problemáticas actuales
\item Detalles de implementación
\item Estado del arte
\end{itemize}
\end{itemize}
\end{frame}

\section{Bibliografía}
\label{sec:org07eaf85}

\begin{frame}[label={sec:orgdfa68e9}]{Fundamentos de Sistemas Operativos}
\begin{columns}\begin{column}{0.35\textwidth}
\begin{center}
\includegraphics[height=15em]{../img/pres/libro_fund_sist_op.jpg}
\end{center}
\end{column}\begin{column}{0.6\textwidth}
\begin{itemize}
\item Libro en el que participé como autor y coordinador
\item \href{http://sistop.org}{Fundamentos de Sistemas Operativos}
\item Con \emph{licenciamiento libre} → ¡Descárguenlo y compártanlo!
\url{https://sistop.org/}
\item Editado por la UNAM
\item Disponible en la ventanilla de apuntes de la Facultad
\item Diseñado siguiendo el plan de estudios de la FI-UNAM
\end{itemize}
\end{column}\end{columns}
\end{frame}

\begin{frame}[label={sec:orge772a61}]{Bibliografía oficial del curso}
\begin{columns}\begin{column}{0.4\textwidth}

\begin{center}
\includegraphics[width=.9\linewidth]{../img/pres/libro_silberschatz.png}
\end{center}
\vfill
\begin{center}
\includegraphics[width=.9\linewidth]{../img/pres/libro_tanenbaum.png}
\end{center}

\end{column}
\begin{column}{0.5\textwidth}

{\large Operating System Concept Essentials} \\
Abraham Silberschatz, Peter Baen Galvin, Greg Gagne\\
Wiley (Traducción: Limusa)\\
{\scriptsize 5ª edición (1998) en adelante }
\vskip 2em
{\large Sistemas operativos: Diseño e implementación} \\
Andrew S. Tanenbaum y Albert S. Woodhull \\
{\scriptsize Prentice Hall ed. 2ª (1997) o 3ª (2006)}

\end{column}\end{columns}
\end{frame}

\begin{frame}[label={sec:orgc4a815d}]{Otros textos recomendados}
\Large An operating systems vade mecum \normalsize \\
Raphael Finkel\\
University of Kentucky - Lexington, 1988\\
\href{ftp://ftp.cs.uky.edu/cs/manuscripts/vade.mecum.2.pdf}{Disponible en línea} desde \href{http://www.cs.uky.edu/\~raphael/}{el sitio Web del autor}

\vskip 2em {\large Operating Systems: Three Easy Pieces } \\
Remzi H. Arpaci-Dusseau y Andrea C. Arpaci-Dusseau \\
University of Wisconsin Madison \\

Disponible en línea desde \href{http://pages.cs.wisc.edu/\~remzi/OSTEP}{el sitio Web del autor} capítulo por
capítulo, y a la venta tanto \href{http://pages.cs.wisc.edu/\~remzi/OSTEP/book-softcover.html}{impreso} como \href{http://pages.cs.wisc.edu/\~remzi/OSTEP/book-electronic.html}{en un sólo PDF}.
\end{frame}

\begin{frame}[label={sec:org76f1c45}]{Pero\ldots{} ¡Toma nota!}
\begin{itemize}
\item De todos modos, \emph{sugiero fuertemente que cada quién tome notas}
\begin{itemize}
\item Ayuda al proceso de aprendizaje
\item Siempre entenderás mejor las cosas en tus propias palabras
\item No está sujeto a que se \emph{caiga} mi servidor
\end{itemize}
\end{itemize}
\end{frame}

\begin{frame}[label={sec:orgb2ee617}]{Sitio Web de la materia}
\begin{center}
\url{http://gwolf.sistop.org/}

Encontrarás:
\end{center}
\begin{itemize}
\item Listas de asistencia, calificaciones
\item Temas sugeridos para las exposiciones (las abordamos en breve)
\item Liga al material \emph{formal}
\end{itemize}
\end{frame}

\begin{frame}[label={sec:org53282de}]{Depósito en GitHub}
\begin{itemize}
\item Entrega de tareas, prácticas, exposiciones y proyectos
\emph{exclusivamente} mediante el \emph{depósito Git}, en la plataforma
\emph{GitHub}
\item Entorno de desarrollo colaborativo
\item Referente para el desarrollo de software (millones de proyectos activos)
\end{itemize}
\begin{center}
\url{http://github.com/unamfi/sistop-2023-1}
\end{center}
\end{frame}

\section{Normas del grupo}
\label{sec:orga36a6b3}
\begin{frame}[label={sec:orgb109a5a}]{Criterios de evaluación}
\begin{center}
\begin{tabular}{ll}
\hline
Proyectos y exámenes & 60\%\\
Tareas y ejercicios & 30\%\\
Exposiciones & 20\%\\
\hline
Máximo posible & 110\%\\
\hline
\end{tabular}
\end{center}
\begin{itemize}
\item Exención de examen ordinario con calificación global de 8.5 y 100\%
de tareas entregadas
\begin{itemize}
\item Si la mayor parte del semestre se desarrolla a distancia,
posiblemente reevaluaré los requisitos
\end{itemize}
\item En caso de obtener más del 100\% \emph{final}, la calificación se recorta
al máximo definido
\end{itemize}
\end{frame}

\begin{frame}[label={sec:org98d8929}]{Acerca de la evaluación}
\begin{center}
Compromiso:

¡Entregar tareas, trabajos y exámenes corregidos a la brevedad!
\end{center}
\begin{itemize}
\item \uline{Intento} tener todo calificado al martes inmediato siguiente a su
aplicación
\begin{itemize}
\item O, por lo menos, explicar claramente si no puedo entregarlos a
tiempo (¡y apurarme!)
\end{itemize}
\item Tareas, proyectos y exámenes parciales son para aprender. \emph{Debo
entregarlos explicando cualquier error}.
\item Debo (por reglamento) conservar los exámenes finales. Les haré
llegar copias en forma electrónica tan pronto termine de
calificarlos.
\end{itemize}
\end{frame}

\begin{frame}[label={sec:orgf2e75da}]{Proyectos}
\begin{itemize}
\item Desarrollo \emph{propio y original} que integre el contenido visto en
determinadas unidades
\begin{itemize}
\item Introducción / Relación con HW
\item Administración / planificación de procesos
\item Sistemas de archivos
\end{itemize}
\item La fecha de entrega será anunciada con por lo menos una semana de
antelación
\item Desarrollados de forma individual o en equipos de dos integrantes
\end{itemize}
\end{frame}

\begin{frame}[label={sec:org6153796}]{Tareas \emph{obligatorias}}
\begin{itemize}
\item La entrega de tareas es \emph{obligatoria}
\begin{itemize}
\item Las tareas se consideran entregadas \emph{el día indicado, o antes si
les resulta imposible}
\item Tareas entregadas fuera de tiempo \alert{no tienen derecho a
calificación} (aunque sí a contar como entregadas, y a revisión de
conceptos)
\end{itemize}
\item Requisito para la exención: \alert{80\%} de tareas
\item Requisito para presentar examen final en primera vuelta: \alert{60\%} de tareas
\end{itemize}
\end{frame}

\begin{frame}[label={sec:orgfb1dc7c}]{Prácticas}
\begin{itemize}
\item Auxiliares para dominar herramientas o conceptos específicos
\item Se califican dentro del rubro de \emph{tareas}
\item Son opcionales, se \emph{agregan} a la calificación obtenida
\begin{itemize}
\item Cuatro prácticas \(\Rightarrow\) una tarea con 10
\end{itemize}
\end{itemize}
\pause
\begin{center}
\alert{¡La práctica 1 está lista para que la hagas!}

Entrega: 2022.08.23
\end{center}
\end{frame}


\begin{frame}[label={sec:org58a822f}]{Exposiciones / Proyectos de investigación}
\begin{itemize}
\item Las exposiciones deben buscar anclar la teoría que estamos
estudiando (o temas relacionados) con la actualidad en el campo
\item Pueden desarrollarse de forma individual o en equipo de 2 personas
\item Expondrán sus proyectos ante el grupo o presentarán a sus alumnos
videos exponiendo sus temas (dependiendo de la \emph{realidad})
\begin{itemize}
\item Exposiciones de \(\approx\) 15 minutos
\item Sesión breve de preguntas durante la clase
\end{itemize}
\item En el sitio Web de la materia hay una lista de temas \emph{sugeridos}
\begin{itemize}
\item Pero vale más (incluso en calificación) el desarrollo de temas de
su interés, con un enfoque original
\end{itemize}
\end{itemize}
\end{frame}

\begin{frame}[label={sec:org1ae1665}]{Autoría de lo que me entreguen}
\begin{itemize}
\item \alert{Todo fraude puede hasta causar baja}
\begin{itemize}
\item Dependiendo de la situación; aplica en tareas, proyectos, exámenes,
ejercicios\ldots{}
\item ¿Trabajo de investigación? ¡Evítense problemas! Nada de \emph{copiar y
pegar}.
\begin{itemize}
\item Es \emph{muy fácil} detectar el copiado-y-pegado
\item Lo hago frecuentemente.
\end{itemize}
\item ¿Repositorio de semestres anteriores de la materia?
\begin{itemize}
\item Pueden usarlo para \emph{inspiración}, para obtener ideas
\item Es \emph{trivial} detectar código copiado\ldots{}
\item No lo hagan. Si copian un fragmento, citen/den atribución
correctamente.
\end{itemize}
\end{itemize}
\end{itemize}
\end{frame}

\begin{frame}[label={sec:orge1fcc9e}]{Toma de asistencia}
\begin{itemize}
\item \alert{Se tomará asistencia} cada sesión. La asistencia a clases \alert{es
obligatoria}.
\end{itemize}

\begin{itemize}
\item Sólo tendrán derecho a exención los alumnos con \alert{80\%} de asistencia
\begin{itemize}
\item Derecho a examen ordinario en primera vuelta los alumnos con \alert{70\%}
de asistencia.
\end{itemize}
\end{itemize}
\end{frame}

\begin{frame}[label={sec:org0319b4f}]{Normas de convivencia}
\begin{itemize}
\item Respeto mutuo ante todo — Incluyendo a uno mismo y al grupo
\begin{itemize}
\item No es lo mismo que trato \emph{formal}.
\item Me parece perfecto que nos \emph{tuteemos} si se sienten cómodos (o no,
si no ;-) )
\end{itemize}
\item Lenguaje correcto, entrar y salir sin estorbar, etc.
\item \sout{Comida en clase: No deseable, pero permitida a condición de que no
hagan ruido, no moleste a terceros (¡olores!) y no dejen basura.}
\begin{itemize}
\item Este semestre, por favor sin comida\ldots{}
\end{itemize}
\item Este semestre \alert{todavía tenemos que usar cubrebocas}, todo el tiempo
que estemos en el salón.
\item Me reservo el derecho a amonestar faltas de respeto \emph{entre
alumnos} según amerite.
\end{itemize}
\end{frame}

\begin{frame}[label={sec:org92c9816}]{Quedo a sus órdenes}
\begin{center}
Repito mis coordenadas:
\end{center}
\begin{description}
\item[{Nombre}] Gunnar Eyal Wolf Iszaevich
\item[{E-mail}] sistop@gwolf.org
\item[{Ubicación}] Instituto de Investigaciones Económicas UNAM
(Secretaría Técnica)
\item[{Teléfono}] \sout{5623-0154 (IIEc-UNAM)}
\item[{Página del curso}] \url{http://gwolf.sistop.org/}
\item[{Repositorio}] \small \url{https://github.com/unamfi/sistop-2020-2/}
\end{description}
\end{frame}
\end{document}