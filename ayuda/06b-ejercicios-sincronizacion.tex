% Created 2023-02-06 Mon 12:04
% Intended LaTeX compiler: pdflatex
\documentclass[presentation]{beamer}
\usepackage[utf8]{inputenc}
\usepackage[T1]{fontenc}
\usepackage{graphicx}
\usepackage{longtable}
\usepackage{wrapfig}
\usepackage{rotating}
\usepackage[normalem]{ulem}
\usepackage{amsmath}
\usepackage{amssymb}
\usepackage{capt-of}
\usepackage{hyperref}
\usepackage[spanish]{babel}
\usepackage{listings}
\pgfdeclareimage[height=0.7\textheight]{../img/pres/cintillo.png}{../img/pres/cintillo.png}\logo{\pgfuseimage{../img/pres/cintillo.png}}
\AtBeginSection[]{ \begin{frame}<beamer> \frametitle{Índice} \tableofcontents[currentsection] \end{frame} }
\definecolor{string}{rgb}{0,0.6,0} \definecolor{shadow}{rgb}{0.5,0.5,0.5} \definecolor{keyword}{rgb}{0.58,0,0.82} \definecolor{identifier}{rgb}{0,0,0.7}
\renewcommand{\ttdefault}{pcr}
\lstset{basicstyle=\ttfamily\scriptsize\bfseries, showstringspaces=false, keywordstyle=\color{keyword}, stringstyle=\color{string}, identifierstyle=\color{identifier}, commentstyle=\mdseries\textit, inputencoding=utf8, extendedchars=true, breaklines=true, breakatwhitespace=true, breakautoindent=true, numbers=left, numberstyle=\ttfamily\tiny\textit}
\newcommand{\rarrow}{$\rightarrow$\hskip 0.5em}
\usetheme{Warsaw}
\usecolortheme{lily}
\author{Gunnar Wolf}
\date{}
\title{Administración de procesos: Ejercicios de sincronización}
\hypersetup{
 pdfauthor={Gunnar Wolf},
 pdftitle={Administración de procesos: Ejercicios de sincronización},
 pdfkeywords={},
 pdfsubject={},
 pdfcreator={Emacs 28.2 (Org mode 9.5.5)}, 
 pdflang={Spanish}}
\begin{document}

\maketitle

\section{Introducción}
\label{sec:org2b1dd9e}

\begin{frame}[label={sec:orgd3d6c0a}]{Introducción}
\begin{itemize}
\item En estas láminas presento varios ejercicios de sincronización
\item Sin un órden claro
\begin{itemize}
\item Simplemente son ejercicios que, comprendiendo los fundamentos de
concurrencia y sincronización, deben poder realizar.
\end{itemize}
\item Sugiero su implementación empleando semáforos
\begin{itemize}
\item Pero pueden emplear cualquier otro mecanismo de sincronización.
\end{itemize}
\item Uso probable de estas láminas
\begin{itemize}
\item Tarea
\item Ejercicios en clase
\end{itemize}
\end{itemize}
\end{frame}

\section{De gatos y ratones}
\label{sec:orgc48bf05}
\begin{frame}[label={sec:org985144f}]{Planteamiento}
\begin{center}
Tengo \(k\) gatos (y desafortunadamente, \(l\) ratones) en su casa.
\vskip 1em
Le sirvo la comida a mis gatos en \(m\) platos.
\vskip 1em
Gatos y ratones han llegado a un acuerdo para repartirse el tiempo y
comida — Pero tienen que convencerme de que están haciendo su trabajo.
\end{center}
\end{frame}

\begin{frame}[label={sec:orgd9a2693}]{Reglas}
\begin{itemize}
\item Los gatos pueden comer de sus \(m\) platos de comida.
\item Los ratones pueden comer de esos mismos platos siempre y cuando no
sean vistos.
\item Si un gato ve a un ratón comiendo, se lo \emph{debe} comer (para
mantener su reputación)
\end{itemize}
\begin{itemize}
\item Los platos están puestos uno junto al otro
\begin{itemize}
\item Sólo un animal puede comer de cada plato a la vez
\item Si un gato está comiendo y un ratón comienza a comer de otro
plato, el gato lo ve (y se lo come).
\item Por \emph{acuerdo de caballeros}, los gatos no se van a acercar a los
platos mientras hay ratones comiendo.
\end{itemize}
\item Importante: ¡Hay que evitar la inanición!
\end{itemize}
\end{frame}

\section{Intersección de caminos}
\label{sec:org87fa391}
\begin{frame}[label={sec:orgada327c}]{Planteamiento}
\begin{center}
Hay un cruce de caminos sin señalamiento vial:

\begin{center}
\includegraphics[height=0.3\textheight]{../img/cruce_caminos.png}
\end{center}

El tránsito puede llegar desde cualquier lugar y en cualquier
momento. ¿Cómo aseguramos que no haya choques?
\end{center}
\end{frame}

\begin{frame}[label={sec:orgd81ac98}]{Reglas}
\begin{itemize}
\item No puede haber dos autos en la misma sección de la intersección a la
vez (llamemos a esa situación \emph{accidente} o \emph{choque})
\item No existe el rebase, los autos no invaden el carril izquierdo.
\item No debes permitir que se lleve a la inanición: Aunque haya tráfico
constante en un sentido, un auto que llegue desde otro debe poder
cruzar.
\end{itemize}
\end{frame}

\begin{frame}[label={sec:org2b79d88}]{Refinamiento 1}
\begin{center}
Tal vez elegiste bloquear la intersección completa cuando un auto
llega.
\end{center}

\begin{itemize}
\item ¿Por qué es ineficiente?
\item ¿Cómo podrías mejorar el rendimiento (reducir la inanición) y
mantener la garantía de no-choques?
\end{itemize}
\end{frame}

\begin{frame}[label={sec:org8710bde}]{Refinamiento 2}
\begin{center}
¿Cómo ajustarías el código para modelar también los giros?
\end{center}
\begin{itemize}
\item Un auto podría girar a la derecha (y emplear sólo un cuadrante)
\item Podría seguir de frente (y emplear dos cuadrantes)
\item Podría girar a la izquierda (y emplear tres cuadrantes).
\item Ojo: ¿Puedes evitar los bloqueos mutuos?
\end{itemize}
\end{frame}

\section{El elevador}
\label{sec:org677cf76}
\begin{frame}[label={sec:org3fd955c}]{Planteamiento}
\begin{itemize}
\item El elevador de la Facultad se descompone demasiado, porque sus
usuarios no respetan los límites.
\begin{itemize}
\item Te toca evitar este desgaste (y el peligro que conlleva).
\end{itemize}
\item Implementa el elevador como un hilo, y a cada persona que quiere
usarlo como otro hilo.
\item El elevador de la Facultad de Ingeniería da servicio a cinco
pisos.
\begin{itemize}
\item Un usuario puede llamarlo en cualquiera de ellos
\item Puede querer ir a cualquiera otro de ellos.
\end{itemize}
\end{itemize}
\end{frame}

\begin{frame}[label={sec:org5edae39}]{Reglas}
\begin{itemize}
\item El elevador tiene capacidad para cinco pasajeros
\begin{itemize}
\item Recuerden que el peso canónico de todo ingeniero (estudiante o
docente) es constante.
\end{itemize}
\item Para ir del piso \(x\) a \(y\), el elevador tiene que cruzar todos los
pisos intermedios
\item Los usuarios prefieren esperar dentro del elevador que fuera de él
\begin{itemize}
\item Si el elevador va subiendo y pasa por el piso \(x\), donde está \(A\)
esperando para bajar, \(A\) aborda al elevador (no espera a que vaya
en la dirección correcta).
\end{itemize}
\end{itemize}
\end{frame}

\begin{frame}[label={sec:org19f0040}]{Refinamiento}
\begin{itemize}
\item Evita la inanición.
\begin{itemize}
\item ¿Cómo puedes asegurarte de que una serie de alumnos que van entre
dos pisos no van a monopolizar al elevador ante otro usuario que
va para otro piso?
\end{itemize}
\end{itemize}
\end{frame}

\section{Los alumnos y el asesor}
\label{sec:org5ca146a}
\begin{frame}[label={sec:org671c791}]{Planteamiento}
\vfill
Un profesor de la facultad asesora a varios estudiantes, y estamos en
su horario de atención.
\vfill
Modelar la interacción durante este horario de modo que la espera
(para todos) sea tan corta como sea posible.
\end{frame}

\begin{frame}[label={sec:orgc37392f}]{Reglas}
\begin{itemize}
\item Un profesor tiene \(x\) sillas en su cubículo
\begin{itemize}
\item Cuando no hay alumnos que atender, las sillas sirven como sofá, y
el profesor se acuesta a dormir la siesta.
\end{itemize}
\item Los alumnos pueden tocar a su puerta en cualquier momento, pero no
pueden entrar más de \(x\) alumnos
\item Para evitar confundir al profesor, sólo un alumno puede presentar su
duda (y esperar a su respuesta) al mismo tiempo.
\begin{itemize}
\item Los demás alumnos sentados deben esperar pacientemente su turno.
\item Cada alumno puede preguntar desde 1 y hasta \(y\) preguntas
(permitiendo que los demás alumnos pregunten entre una y otra)
\end{itemize}
\end{itemize}
\end{frame}

\section{El servidor Web}
\label{sec:orgffff576}
\begin{frame}[label={sec:org350e633}]{Planteamiento}
\begin{center}
Al presentar los modelos de programación con hilos presentamos al
\emph{Jefe-trabajador}:

\begin{center}
\includegraphics[height=0.3\textheight]{../img/dot/jefe_trabajador.png}
\end{center}

Dijimos que así operan muchos servidores de red, como el servidor
Apache.

¿Cómo modelarías la interacción entre jefe y trabajadores, empleando
mecanismos de sincronización?
\end{center}
\end{frame}

\begin{frame}[label={sec:org948bbad}]{Reglas}
\begin{itemize}
\item Al inicializar, el proceso jefe lanza \(k\) hilos trabajadores
\begin{itemize}
\item Los trabajadores que no tienen nada que hacer se van a dormir.
\end{itemize}
\item El proceso jefe recibe una conexión de red, y elige a cualquiera de
los trabajadores para que la atienda
\begin{itemize}
\item Se la asigna a un trabajador y lo despierta
\end{itemize}
\item El jefe va a buscar mantener siempre a \(k\) hilos disponibles y
listos para atender las solicitudes que van llegando.
\end{itemize}
\end{frame}

\begin{frame}[label={sec:org71ed05e}]{Refinamiento}
\begin{itemize}
\item Seguimiento del sistema: ¿Cómo implementarías lo necesario para
mantener \emph{información de contabilidad}?
\begin{itemize}
\item Cada hilo debe notificar antes de terminar su ejecución,
entregando información de su rendimiento
\item Por ejemplo, qué página fue solicitada
\end{itemize}
\end{itemize}
\end{frame}


\section{El cruce del río}
\label{sec:org346bdba}
\begin{frame}[label={sec:orged8c50a}]{Planteamiento}
\begin{itemize}
\item Para llegar a un encuentro de desarrolladores de sistemas
operativos, hace falta cruzar un río en balsa.
\item Los desarrolladores podrían pelearse entre sí, hay que cuidar que
vayan con un balance adecuado
\end{itemize}
\end{frame}

\begin{frame}[label={sec:org29911b1}]{Reglas}
\begin{itemize}
\item En la balsa caben cuatro (y sólo cuatro) personas
\begin{itemize}
\item La balsa es demasiado ligera, y con menos de cuatro puede volcar.
\end{itemize}
\item Al encuentro están invitados \emph{hackers} (desarrolladores de Linux) y
\emph{serfs} (desarrolladores de Microsoft).
\begin{itemize}
\item Para evitar peleas, debe mantenerse un buen balance: No debes
permitir que aborden tres \emph{hackers} y un \emph{serf}, o tres \emph{serfs} y
un \emph{hacker}. Pueden subir cuatro del mismo \emph{bando}, o dos y dos.
\end{itemize}
\item Hay sólo una balsa.
\item No se preocupen por devolver la balsa (está programada para volver
sola)
\end{itemize}
\end{frame}

\section{Santa Claus}
\label{sec:orga931b56}
\begin{frame}[label={sec:org67da581}]{Planteamiento}
\begin{itemize}
\item Santa Claus duerme en el Polo Norte mientras sus elfos trabajan
frenéticamente en la construcción de millones de nuevos juguetes
\begin{itemize}
\item A veces se topan con un problema — Pueden pedir ayuda a Santa
Claus, pero sólo de tres en tres.
\end{itemize}
\item Sus nueve renos pasan todo el año de vacaciones en las playas del
Caribe
\begin{itemize}
\item Santa debe despertar a tiempo para iniciar su viaje anual
\end{itemize}
\end{itemize}
\end{frame}

\begin{frame}[label={sec:orgb98f88c}]{Reglas}
\begin{itemize}
\item Si los nueve renos están de vuelta, es hora de despertar a Santa
Claus para que inicie su recorrido.
\item Si los elfos tienen problemas construyendo algún juguete, le piden
ayuda a Santa Claus
\begin{itemize}
\item Pero para no darle demasiada lata, lo hacen sólo cuando hay tres
elfos que tienen un problema. Mientras tanto, lo dejan dormir.
\item Puede haber cualquier cantidad de elfos.
\end{itemize}
\end{itemize}
\end{frame}
\end{document}