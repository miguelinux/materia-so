% Created 2023-02-06 Mon 12:04
% Intended LaTeX compiler: pdflatex
\documentclass[presentation]{beamer}
\usepackage[utf8]{inputenc}
\usepackage[T1]{fontenc}
\usepackage{graphicx}
\usepackage{longtable}
\usepackage{wrapfig}
\usepackage{rotating}
\usepackage[normalem]{ulem}
\usepackage{amsmath}
\usepackage{amssymb}
\usepackage{capt-of}
\usepackage{hyperref}
\usepackage[spanish]{babel}
\usepackage{listings}
\pgfdeclareimage[height=0.7\textheight]{../img/pres/cintillo.png}{../img/pres/cintillo.png}\logo{\pgfuseimage{../img/pres/cintillo.png}}
\AtBeginSection[]{ \begin{frame}<beamer> \frametitle{Índice} \tableofcontents[currentsection] \end{frame} }
\definecolor{string}{rgb}{0,0.6,0} \definecolor{shadow}{rgb}{0.5,0.5,0.5} \definecolor{keyword}{rgb}{0.58,0,0.82} \definecolor{identifier}{rgb}{0,0,0.7}
\renewcommand{\ttdefault}{pcr}
\lstset{basicstyle=\ttfamily\scriptsize\bfseries, showstringspaces=false, keywordstyle=\color{keyword}, stringstyle=\color{string}, identifierstyle=\color{identifier}, commentstyle=\mdseries\textit, inputencoding=utf8, extendedchars=true, breaklines=true, breakatwhitespace=true, breakautoindent=true, numbers=left, numberstyle=\ttfamily\tiny\textit}
\newcommand{\rarrow}{$\rightarrow$\hskip 0.5em}
\usetheme{Warsaw}
\usecolortheme{lily}
\author{Gunnar Wolf}
\date{}
\title{Sistemas de archivos: El medio físico}
\hypersetup{
 pdfauthor={Gunnar Wolf},
 pdftitle={Sistemas de archivos: El medio físico},
 pdfkeywords={},
 pdfsubject={},
 pdfcreator={Emacs 28.2 (Org mode 9.5.5)}, 
 pdflang={Spanish}}
\begin{document}

\maketitle

\section{Detalles del medio magnético}
\label{sec:orga3da7bc}

\begin{frame}[label={sec:org9511360}]{Empleando discos duros: \emph{Notación C-H-S}}
\begin{itemize}
\item A lo largo de los últimos 40 años, el principal medio de
almacenamiento ha sido el \emph{disco duro}
\item Para hacer referencia a un sector específico de datos, la notación
tradicional empleada es la C-H-S (\emph{Cilindro - Cabeza - Sector})
\item Permite referir a cualquier punto del disco dentro de un espacio
tridimensional
\end{itemize}
\end{frame}

\begin{frame}[label={sec:org7909de6}]{Mapeo de un disco duro a \emph{C-H-S}}
\begin{figure}[htbp]
\centering
\includegraphics[height=0.7\textheight]{../img/cilindro_cabeza_sector.png}
\caption{Coordenadas de un disco duro, presentando cada uno de sus sectores en C-H-S (Silberschatz, p.458)}
\end{figure}
\end{frame}

\begin{frame}[label={sec:org972de4c}]{Algoritmos de planificación de acceso a disco}
\begin{itemize}
\item Si el disco es la parte más lenta de un sistema de cómputo, vale la
pena dedicar tiempo a encontrar el mejor ordenamiento posible para
lecturas y escrituras
\item Veremos algunos de los algoritmos históricos
\begin{itemize}
\item Como referencia
\item Para comparar sus puntos de partida
\end{itemize}
\item Pero no profundizaremos mucho al respecto — Estos esquemas \emph{ya no se emplean}
\begin{itemize}
\item Fuera del desarrollo de controladores embebidos
\item Veremos también las razones para su abandono
\end{itemize}
\item Trabajaremos partiendo del cilindro 53, con la cadena de referencia
\emph{98, 183, 37, 122, 14, 124, 65, 67}
\end{itemize}
\end{frame}

\begin{frame}[label={sec:org82c3fe0}]{Acceso a disco en FIFO}
\begin{itemize}
\item Como en los otros subsistemas que hemos visto, el primer algoritmo
es atender a las solicitudes \emph{en órden de llegada}
\item Algoritmo \emph{justo}, aunque poco eficiente
\item Movimiento total de cabezas para la cadena de referencia: 640
cilindros
\begin{itemize}
\item Con movimientos tan aparentemente absurdos como 122 \rarrow 14
\rarrow 124
\end{itemize}
\end{itemize}
\end{frame}

\begin{frame}[label={sec:org327f465}]{Tiempo más corto a continuación (SSTF)}
\begin{itemize}
\item \emph{Shortest Seek Time First} — Corresponde conceptualmente a \emph{Shortest
Job First} (de \emph{planificación de procesos})
\item Reduce el desplazamiento total a partir de FIFO de 640 a sólo 236 cilindros
\item Puede llevar a la inanición
\begin{itemize}
\item Al favorecer a las solicitudes cercanas, las lejanas pueden quedar
a la espera indefinidamente
\end{itemize}
\end{itemize}
\end{frame}

\begin{frame}[label={sec:org5223973}]{Acceso a disco en elevador (SCAN)}
\begin{itemize}
\item Evita la inanición, buscando minimizar el movimiento de las cabezas
\item Opera como elevador: La cabeza recorre el disco de extremo a extremo
\begin{itemize}
\item Atiende a todas las solicitudes que haya pendientes \emph{en el camino}
\end{itemize}
\item Los recorridos pueden ser mayores a SSTF
\item Pero garantíza que no habrá inanición
\begin{itemize}
\item En este recorrido en particular, también 236 cilindros (iniciando
en 53 y \emph{hacia abajo})
\end{itemize}
\item Modificación menor que mejora el rendimiento: LOOK
\begin{itemize}
\item Verificar si hay algún otro sector pendiente en la dirección
actual; si no, dar la vuelta anticipadamente
\item Reduciría el recorrido a 208 cilindros
\end{itemize}
\end{itemize}
\end{frame}

\begin{frame}[label={sec:orgaf5cf19}]{Comparación de los algoritmos}
\begin{figure}[htbp]
\centering
\includegraphics[width=0.9\textwidth]{../img/gnuplot/mov_cabeza_por_algoritmo.png}
\caption{Movimientos de las cabezas bajo los diferentes algoritmos planificadores de acceso a disco, indicando la distancia total recorrida por la cabeza bajo cada uno, iniciando con la cabeza en la posición 53. Para SCAN, LOOK y C-SCAN, asumimos que iniciamos con la cabeza en dirección decreciente.}
\end{figure}
\end{frame}

\begin{frame}[label={sec:orga05ac56}]{¿Por qué ya no se emplean estos algoritmos?}
\begin{itemize}
\item Requieren más información de la disponible
\begin{itemize}
\item Están orientados a reducir el traslado \emph{de la cabeza}
\item Ignoran la \emph{demora rotacional}
\item La demora rotacional va entre \(1 \over 10\) y \(1 \over 3\) del
tiempo de traslado de cabezas
\end{itemize}
\item Distintas prioridades para distintas solicitudes
\begin{itemize}
\item Si el sistema operativo prefiere priorizar expresamente, estos
algoritmos no ofrecen la \emph{expresividad} necesaria
\item Por ejemplo, acceso a memoria virtual sobre acceso a archivos
\end{itemize}
\item Abstracciones a niveles más bajos (p.ej. LBA, que veremos a
continuación)
\item Dispositivos \emph{virtuales}
\end{itemize}
\end{frame}

\begin{frame}[label={sec:org7303e43}]{La transición a \emph{LBA}}
\begin{itemize}
\item C-H-S impone muchas restricciones al acomodo de la información
\begin{itemize}
\item No permite mapearse naturalmente a dispositivos que no sean discos
rotativos
\item Hacia principios de los 1990, el \emph{BIOS} imponía límites
innecesarios al almacenamiento (p.ej. número máximo de cilindros)
\end{itemize}
\item Los controladores de disco comenzaron a exponer al sistema una
dirección \emph{lineal}: \emph{Direccionamiento Lógico de Bloques} (\emph{Logical
Block Addressing}, \emph{LBA})
\begin{itemize}
\item Ya no \emph{tridimensional}
\end{itemize}
\item \(LBA = ((C \times HPC) + H) \times SPT + S - 1\)
\begin{itemize}
\item \(HPC\) = cabezas por cilindro
\item \(SPT\) = sectores por pista
\end{itemize}
\end{itemize}
\end{frame}

\begin{frame}[label={sec:orgdbb940b}]{LBA y la reubicación}
\begin{itemize}
\item Sistema operativo y aplicaciones ya sólo hacen referencia por esta
ubicación, no conocen las ubicaciones físicas
\item LBA permite al controlador de disco utilizar más eficientemente el
espacio
\begin{itemize}
\item Número de sectores por track variable
\end{itemize}
\item Responder \emph{preventivamente} a fallos en el medio físico
\begin{itemize}
\item Reubicar sectores \emph{difíciles de leer} antes de que presenten
pérdida de datos
\item Diferentes algoritmos de reubicación \rarrow Mantener tanto como
se pueda el mapeo de los bloques contiguos a ojos del sistema
\end{itemize}
\end{itemize}
\end{frame}

\section{Estado sólido}
\label{sec:orga052045}

\begin{frame}[label={sec:org4ac8fcf}]{Medios de \emph{estado sólido}}
\begin{itemize}
\item Desde hace cerca de una década va creciendo consistentemente el uso
de medios de almacenamiento de \emph{estado sólido}
\begin{itemize}
\item Medios \emph{sin partes móviles}
\end{itemize}
\item Las unidades de estado sólido tienen características muy distintas
a las de los \emph{discos}
\begin{itemize}
\item Pero mayormente seguimos empleando los mismos sistemas de archivos
\end{itemize}
\item Las métricas de confiabilidad y rendimiento tienen que ser replanteadas
\item Un claro espacio de investigación e implementación actual
\end{itemize}
\end{frame}

\begin{frame}[label={sec:org4e41d88}]{Emulación de disco: ¿Un acierto o un error?}
\begin{itemize}
\item Casi todos los discos de estado sólido se presentan al sistema
operativo como un disco \emph{estándar}
\item Ventajas:
\begin{itemize}
\item Permite que sean empleados sin cambios al sistema operativo
\item No hay que pensar en controladores específicos
\end{itemize}
\item Desventaja
\begin{itemize}
\item No se aprovechan sus características únicas
\item Se tienen que adecuar a las restricciones (artificiales) de
sistemas pensados para medios rotativos
\end{itemize}
\end{itemize}
\end{frame}

\begin{frame}[label={sec:org907cdce}]{Estado sólido basado en RAM}
\begin{columns} \begin{column}{0.4\textwidth}
\begin{figure}[htbp]
\centering
\includegraphics[width=\textwidth]{../img/estado_solido_ddr_drivex1.jpg}
\caption{Unidad de estado sólido basado en RAM: DDRdrive X1 (\href{https://en.wikipedia.org/wiki/Solid-state\\\_drive}{Imagen: Wikipedia})}
\end{figure}
\end{column} \begin{column}{0.6\textwidth}
\begin{itemize}
\item Primeros discos de estado sólido
\item Memoria RAM estándar, con una batería de respaldo
\item Extremadamente rápido (velocidad igual a la de acceso a memoria)
\item Muy caro
\item Riesgo de pérdida de datos si se acaba la batería
\end{itemize}
\end{column} \end{columns}
\end{frame}

\begin{frame}[label={sec:orgf221bb3}]{Estado sólido basado en Flash con interfaz SATA}
\begin{columns} \begin{column}{0.4\textwidth}
\begin{figure}[htbp]
\centering
\includegraphics[width=\textwidth]{../img/estado_solido_sata.jpg}
\caption{Unidad de estado sólido basado en Flash con interfaz SATA (\href{https://en.wikipedia.org/wiki/Solid-state\\\_drive}{Imagen: Wikipedia})}
\end{figure}
\end{column} \begin{column}{0.6\textwidth}
\begin{itemize}
\item Se comporta como un disco duro estándar
\item Usos principales:
\begin{itemize}
\item Industria aeroespacial, militar (desde \(\sim 1995\))
\begin{itemize}
\item Por su muy alta resistencia a la vibración
\end{itemize}
\item Servidores, alto rendimiento
\begin{itemize}
\item Ciertas tecnologías presentan velocidad muy superior a la de
medios rotativos
\end{itemize}
\item Presente en \emph{netbooks}, dispositivos móviles (integrado a la
tarjeta madre)
\end{itemize}
\end{itemize}
\end{column} \end{columns}
\end{frame}


\begin{frame}[label={sec:org9e9b710}]{Estado sólido Flash para \emph{transporte temporal}}
\begin{columns} \begin{column}{0.4\textwidth}
\begin{figure}[htbp]
\centering
\includegraphics[width=\textwidth]{../img/estado_solido_usb.png}
\caption{Unidad de estado sólido basado en Flash con interfaz USB (\href{https://en.wikipedia.org/wiki/Solid-state\\\_drive}{Imagen: Wikipedia})}
\end{figure}
\end{column} \begin{column}{0.6\textwidth}
\begin{itemize}
\item Mecanismo de transporte de archivos personales
\item Muy bajo costo
\begin{itemize}
\item Muchos modelos con calidad deficiente
\end{itemize}
\item Muy distintos modelos de conector
\begin{itemize}
\item SD, USB, MMC, etc.
\item Mismo tipo de dispositivo
\end{itemize}
\item Muy alta varianza en capacidad, rendimiento y durabilidad según la
generación tecnológica
\end{itemize}
\end{column} \end{columns}
\end{frame}

\begin{frame}[label={sec:org8f0e752}]{Diferencias del medio}
\begin{itemize}
\item Tiempo constante de acceso al medio
\begin{itemize}
\item Desaparece la demora rotacional y de movimiento de brazo
\end{itemize}
\item Tamaño de sector: Típicamente 4KB (ya no 512 bytes, estándar desde
los 1950s)
\begin{itemize}
\item Debería traducirse a una alineación de estructuras — No siempre
es el caso
\end{itemize}
\item Diferencia de velocidad
\begin{itemize}
\item Lectura más rápida, escritura más lenta
\item Ciclos de borrado previos a la escritura
\end{itemize}
\item Desgaste del medio: \(\approx\) miles a cientos de miles de escrituras
\begin{itemize}
\item Nivelamiento de uso (\emph{wear leveling})
\item Efectuado en el controlador (transparente al sistema operativo)
\end{itemize}
\end{itemize}
\end{frame}

\begin{frame}[label={sec:org389bb5a}]{Espacio para mejoramiento / investigación}
\begin{itemize}
\item Esta es un área frustrantemente ignorada
\begin{itemize}
\item Con los diferentes perfiles que requieren los distintos usos de
los medios Flash
\end{itemize}
\item Cada vez más empleada
\item Un buen ámbito para desarrollar proyectos de investigación
\end{itemize}
\end{frame}

\section{Manejo avanzado de volúmenes}
\label{sec:orgd8e3e0c}

\begin{frame}[label={sec:orga09de32}]{¿Diferencia entre \emph{volumen} y \emph{partición}?}
\begin{itemize}
\item En un equipo estándar de escritorio, estos dos términos son
intercambiables
\item Pero es cada vez más frecuente hablar de \emph{arreglos} de discos
\begin{itemize}
\item Alejándonos –por otro camino– del concepto de disco magnético
rotativo estándar
\end{itemize}
\item Principales ejes:
\begin{itemize}
\item Confiabilidad
\item Rendimiento
\item Espacio por volumen
\end{itemize}
\end{itemize}
\end{frame}

\begin{frame}[label={sec:org3b061a3}]{RAID}
\begin{itemize}
\item \emph{Arreglo Redundante de Discos Baratos} (\emph{Redundant Array of
Inexpensive Disks})
\begin{itemize}
\item Ocasionalmente lo encontraremos como de discos \emph{Independientes}
(\emph{Independent})
\end{itemize}
\item \emph{Diferentes} técnicas de combinar un conjunto de discos,
presentándose como uno sólo a \emph{capas superiores}
\item Originalmente (1990s) implementado sólo en el controlador (en
hardware)
\begin{itemize}
\item Hoy lo implementan casi todos los sistemas operativos como opción
\item Diferencia no-nula (pero despreciable) en rendimiento
\end{itemize}
\item Hay muchas alternativas basadas en las \emph{ideas} de RAID
\begin{itemize}
\item Integrando más niveles del almacenamiento
\item JFS+LSM (AIX), LVM (Linux), ZFS (Solaris, *BSD), \ldots{}
\end{itemize}
\end{itemize}
\end{frame}

\begin{frame}[label={sec:org64571aa}]{Niveles de RAID}
\begin{center}
RAID define diferentes niveles de operación; varios de ellos no son
ya empleados hoy en día. Los principales en uso son:
\end{center}
\begin{description}
\item[{0}] Concatenación
\item[{1}] Espejeo
\item[{5}] Paridad dividida por bloques (\emph{block-interleaved parity})
\item[{6}] Paridad por redundancia P+Q
\end{description}
\end{frame}

\begin{frame}[label={sec:org85ddbd8}]{RAID 0: División en \emph{franjas}}
\begin{itemize}
\item Espacio total: Suma de los espacios de todas sus unidades
\item Mejoría en velocidad (más cabezas independientes)
\item Menor confiabilidad: Un fallo en \emph{cualquiera de los discos} del
volumen lleva a pérdida de información
\end{itemize}
\begin{figure}[htbp]
\centering
\includegraphics[width=\textwidth]{../img/dot/raid_0.png}
\caption{Cinco discos organizados en RAID 0}
\end{figure}
\end{frame}

\begin{frame}[label={sec:orge50fcae}]{RAID 1: Espejo}
\begin{itemize}
\item Espacio total: Uno sólo de los volúmenes
\item Ligera degradación en velocidad (debemos esperar a los datos \emph{de
ambos discos} y validar que sean iguales)
\item Confiabilidad: Soporta la pérdida de un disco
\end{itemize}
\begin{figure}[htbp]
\centering
\includegraphics[width=0.6\textwidth]{../img/dot/raid_1.png}
\caption{Dos discos organizados en RAID 1}
\end{figure}
\end{frame}

\begin{frame}[label={sec:org78d0ab9}]{RAID 5: Paridad dividida por bloques}
\begin{itemize}
\item Espacio total: Suma de los espacios de todas sus unidades \emph{menos
una}
\item Reducción en velocidad (todas las unidades deben leer el sector y
recalcular el resultado)
\item El disco de paridad es otro \emph{a cada bloque}
\item Soporta la pérdida de un disco (cualquiera de ellos)
\end{itemize}
\begin{figure}[htbp]
\centering
\includegraphics[width=0.9\textwidth]{../img/dot/raid_5.png}
\caption{Cinco discos organizados en RAID 5}
\end{figure}
\end{frame}

\begin{frame}[label={sec:org81e0813}]{RAID 6: Paridad por redundancia P+Q}
\begin{itemize}
\item Espacio total: Suma de los espacios de todas sus unidades \emph{menos
dos}
\item Reducción en velocidad (todas las unidades deben leer el sector y
recalcular el resultado)
\item El disco de paridad es otro \emph{a cada bloque}
\item Soporta la pérdida de dos discos (cualquiera de ellos)
\end{itemize}
\begin{figure}[htbp]
\centering
\includegraphics[width=0.6\textwidth]{../img/dot/raid_6.png}
\caption{Cinco discos organizados en RAID 6}
\end{figure}
\end{frame}

\begin{frame}[label={sec:orge86b5ea}]{Combinando niveles: RAID 1+0 (o RAID 10)}
\begin{columns}\begin{column}{0.6\textwidth}
\begin{itemize}
\item Pueden combinarse dos niveles para obtener los beneficios de ambos
\item Concatenación de unidades espejeadas
\item Soporta el fallo de hasta el 50\% de los discos
\begin{itemize}
\item Siempre y cuando un disco por grupo se mantenga operando
\end{itemize}
\end{itemize}
\end{column} \begin{column}{0.4\textwidth}
\begin{figure}[htbp]
\centering
\includegraphics[width=\textwidth]{../img/dot/raid_10.png}
\caption{Seis discos organizados en RAID 1+0}
\end{figure}
\end{column}\end{columns}
\end{frame}

\begin{frame}[label={sec:orgbba68ad}]{Combinación inconveniente de RAID: 0+1}
\begin{columns}\begin{column}{0.6\textwidth}
\begin{itemize}
\item El órden de los factores \emph{altera} el producto
\item Espejo de unidades concatenadas
\item Soporta también el fallo de hasta el 50\% de los discos
\begin{itemize}
\item Pero \emph{únicamente} si ocurren en el mismo volumen RAID1
\end{itemize}
\item Obtenemos el mismo resultado \emph{aparente} que RAID 1+0, pero perdemos
confiabilidad
\end{itemize}
\end{column} \begin{column}{0.4\textwidth}
\begin{figure}[htbp]
\centering
\includegraphics[width=\textwidth]{../img/dot/raid_01.png}
\caption{Seis discos organizados en RAID 0+1}
\end{figure}
\end{column}\end{columns}
\end{frame}
\end{document}