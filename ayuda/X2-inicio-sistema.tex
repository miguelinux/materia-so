% Created 2023-02-06 Mon 12:04
% Intended LaTeX compiler: pdflatex
\documentclass[presentation]{beamer}
\usepackage[utf8]{inputenc}
\usepackage[T1]{fontenc}
\usepackage{graphicx}
\usepackage{longtable}
\usepackage{wrapfig}
\usepackage{rotating}
\usepackage[normalem]{ulem}
\usepackage{amsmath}
\usepackage{amssymb}
\usepackage{capt-of}
\usepackage{hyperref}
\usepackage[spanish]{babel}
\usepackage{listings}
\pgfdeclareimage[height=0.7\textheight]{../img/pres/cintillo.png}{../img/pres/cintillo.png}\logo{\pgfuseimage{../img/pres/cintillo.png}}
\AtBeginSection[]{ \begin{frame}<beamer> \frametitle{Índice} \tableofcontents[currentsection] \end{frame} }
\definecolor{string}{rgb}{0,0.6,0} \definecolor{shadow}{rgb}{0.5,0.5,0.5} \definecolor{keyword}{rgb}{0.58,0,0.82} \definecolor{identifier}{rgb}{0,0,0.7}
\renewcommand{\ttdefault}{pcr}
\lstset{basicstyle=\ttfamily\scriptsize\bfseries, showstringspaces=false, keywordstyle=\color{keyword}, stringstyle=\color{string}, identifierstyle=\color{identifier}, commentstyle=\mdseries\textit, inputencoding=utf8, extendedchars=true, breaklines=true, breakatwhitespace=true, breakautoindent=true, numbers=left, numberstyle=\ttfamily\tiny\textit}
\newcommand{\rarrow}{$\rightarrow$\hskip 0.5em}
\usetheme{Warsaw}
\usecolortheme{lily}
\author{Gunnar Wolf}
\date{}
\title{Inicio del sistema}
\hypersetup{
 pdfauthor={Gunnar Wolf},
 pdftitle={Inicio del sistema},
 pdfkeywords={},
 pdfsubject={},
 pdfcreator={Emacs 28.2 (Org mode 9.5.5)}, 
 pdflang={Spanish}}
\begin{document}

\maketitle

\section{Introducción}
\label{sec:orgc7e0c02}
\begin{frame}[label={sec:org94769c8}]{Enfocada a lo específico}
\begin{itemize}
\item Voy a enfocar esta presentación al inicio de un sistema Linux
\begin{itemize}
\item Nos da el mayor detalle acerca del proceso
\item Guarda bitácora de toda la información de inicio
\item Permite \emph{implementaciones competitivas} de subsistemas (y permite
un debate amplio, rico al respecto)
\end{itemize}
\end{itemize}
\end{frame}

\begin{frame}[label={sec:org8e0fb14}]{¿Es tan difícil?}
\begin{itemize}
\item La arquitectura esquemática básica de una computadora es bastante
trivial
\item Cargar un sistema operativo\ldots{} Tal vez no tanto
\item Nos desviaremos brevemente del programa formal de la materia para
comprender qué es lo que ocurre \emph{en los primeros instantes}
\begin{itemize}
\item Buscando entender hasta el punto en que tenemos un sistema
\emph{usable}
\end{itemize}
\end{itemize}
\end{frame}

\section{La carga inicial}
\label{sec:org20d2de9}
\begin{frame}[label={sec:orge212ef7}]{Encendiendo la computadora: Alcance del BIOS}
\begin{itemize}
\item Verificación de \emph{sanidad} del sistema (POST)
\begin{description}
\item[{North Bridge}] CPU+RAM (arquitectura von Neumann)
\item[{South Bridge}] Dispositivos, componentes adicionales (USB, PCI,
red, SATA, etc.)
\end{description}
\item Enumeración básica de dispositivos
\item Selección del dispositivo de arranque
\item Carga del \emph{primer sector} del dispositivo seleccionado
\begin{itemize}
\item Verificación de \emph{integridad} del código a ejecutar (lo veremos a
detalle la siguiente sesión)
\item Transferencia de control a los elementos \emph{provistos por el
usuario} (sistema operativo)
\end{itemize}
\end{itemize}
\end{frame}

\begin{frame}[label={sec:org63782ee}]{¿Por qué tenemos programas \emph{cargadores}?}
\begin{itemize}
\item En arquitecturas derivadas de Intel, el BIOS opera en \emph{modo real}
\begin{itemize}
\item Esto es, con un límite de 640Kb RAM, según la especificación de
la PC de 1981
\item En años recientes, se va migrando de BIOS a EFI \(\rightarrow\)
Veremos la próxima clase lo que eso conlleva
\end{itemize}
\item El conocimiento del BIOS del sistema es muy limitado
\begin{itemize}
\item Su especificación no sabe más que pedir \emph{el primer sector} (512
bytes) del disco duro
\item Cualquier programa que cargue, debe caber en 512 bytes
\item De ahí que tengamos \emph{cargadores de inicio} (\emph{boot loaders})
\end{itemize}
\end{itemize}
\end{frame}

\begin{frame}[label={sec:orgb4f314a}]{Arquitecturas con entornos de inicio \emph{inteligentes}}
\begin{itemize}
\item Históricamente ha habido arquitecturas que han ofrecido entornos de
arranque \emph{inteligentes}
\item Entornos de arranque \emph{programables} (típicamente basadas en Forth)
\begin{itemize}
\item Arquitecturas que lo emplea(ba)n: Sparc, Alpha, PowerPC
\end{itemize}
\item Capacidad de alterar parámetros para el inicio del sistema
\begin{itemize}
\item Especificar parámetros al kernel
\item Elegir un distinto dispositivo de arranque
\item Diagnósticos básicos del sistema
\item Operaciones básicas de red
\end{itemize}
\item Muchas de estas capacidades han ido integrándose al BIOS
\end{itemize}
\end{frame}

\begin{frame}[label={sec:org7062943}]{Los cargadores de inicio}
\begin{itemize}
\item Prácticamente cualquier sistema operativo moderno depende de un
cargador de inicio
\item Estos han ido creciendo para convertirse en verdaderos
mini-sistemas operativos
\begin{itemize}
\item Partcularmente en el área de sistemas de archivo y de E/S
\item Enumeración de dispositivos (no siempre \emph{heredado} del BIOS)
\item Comprensión de distintos sistemas de archivos (incluso
abstracciones como RAID/LVM)
\item Análisis del estado de la última carga (sugiriendo \emph{modo a prueba
de fallos})
\item Edición de los parámetros de invocación del kernel
\end{itemize}
\end{itemize}
\end{frame}

\begin{frame}[label={sec:orge4dfd75}]{El trabajo del cargador}
\begin{center}
Por fin, el trabajo del cargador típicamente se reduce a:
\end{center}
\begin{itemize}
\item Informar al usuario que \emph{todo va bien}
\item Reconocer el entorno
\item Ubicar y cargar la imágen del sistema operativo en el medio de
arranque
\begin{itemize}
\item Posiblemente también de un \emph{disco de inicio mínimo}
\end{itemize}
\item Especificar parámetros de inicio
\item Transferir la ejecución (y \emph{suicidarse})
\end{itemize}
\begin{center}
\ldots{}Y entramos en el terreno del sistema operativo
\end{center}
\end{frame}

\section{Reconociendo el entorno}
\label{sec:org485ee87}
\begin{frame}[label={sec:orgaf0120b},fragile]{La fuente de nuestros datos}
 \begin{itemize}
\item Bitácora en \texttt{/var/log/dmesg}
\item La bitácora tiene un \emph{timestamp} en cada línea de su bitácora, con
resolución de microsegundos
\begin{itemize}
\item Eliminada de lo que aquí muestro para que quepa mejor en pantalla
\end{itemize}
\end{itemize}
\end{frame}

\begin{frame}[label={sec:org8016288},fragile]{Primeros pasos: ¿Dónde estoy?}
 \begin{center}
¿La arquitectura es capaz de correr el núcleo en cuestión?
\end{center}
\begin{verbatim}
Linux version 3.2.0-4-amd64 (debian-kernel@lists.debian.org) (gcc version 4.6.3 (Debian 4.6.3-14) ) #1 SMP Debian 3.2.35-2
Command line: BOOT_IMAGE=/vmlinuz-3.2.0-4-amd64 root=/dev/mapper/mosca-root ro vga=791 quiet splash
\end{verbatim}
\begin{center}
¿Cuál es el \emph{mapa de memoria}?
\end{center}
\begin{verbatim}
BIOS-provided physical RAM map:
 BIOS-e820: 0000000000000000 - 000000000009ec00 (usable)
 BIOS-e820: 00000000000f0000 - 0000000000100000 (reserved)
 BIOS-e820: 0000000000100000 - 00000000cd9ffc00 (usable)
 BIOS-e820: 00000000cd9ffc00 - 00000000cda53c00 (ACPI NVS)
 BIOS-e820: 00000000cda53c00 - 00000000cda55c00 (ACPI data)
 BIOS-e820: 00000000cda55c00 - 00000000d0000000 (reserved)
   (...)
 BIOS-e820: 00000000ffb00000 - 0000000100000000 (reserved)
 BIOS-e820: 0000000100000000 - 0000000128000000 (usable)
\end{verbatim}
\end{frame}

\begin{frame}[label={sec:orge6c693b},fragile]{Características base del equipo}
 \begin{center}
Sigue descubriendo información sobre el CPU y la memoria
\end{center}
\begin{verbatim}
NX (Execute Disable) protection: active
  (...)
found SMP MP-table at [ffff8800000fe710] fe710
ACPI: XSDT 00000000000fc7f0 0008C (v01 DELL    B10K    00000015 ASL  00000061)
  (...)
No NUMA configuration found
Faking a node at 0000000000000000-0000000128000000
Initmem setup node 0 0000000000000000-0000000128000000
  (...)
ACPI: IRQ0 used by override.
ACPI: IRQ2 used by override.
ACPI: IRQ9 used by override.
Using ACPI (MADT) for SMP configuration information
ACPI: HPET id: 0x8086a701 base: 0xfed00000
SMP: Allowing 8 CPUs, 6 hotplug CPUs
  (...)
Booting paravirtualized kernel on bare hardware
\end{verbatim}
\end{frame}

\begin{frame}[label={sec:org693b3fa},fragile]{Inicia la ejecución y recorrida de los \emph{buses}}
 \begin{verbatim}
Memory: 3880132k/4849664k available (3418k kernel code, 825804k absent, 143728k reserved, 3319k data, 576k init)
Hierarchical RCU implementation.
RCU dyntick-idle grace-period acceleration is enabled.
NR_IRQS:33024 nr_irqs:744 16
Console: colour dummy device 80x25
console [tty0] enabled
hpet clockevent registered
Fast TSC calibration using PIT
Detected 2992.557 MHz processor.
\end{verbatim}
\begin{itemize}
\item Hasta este punto, \emph{todo ocurre con t=0.000000}
\item No es que sea un proceso \emph{tan} instantáneo, sino que el kernel no
ha comenzado a registrar el paso del tiempo
\end{itemize}
\begin{verbatim}
Calibrating delay loop (skipped), value calculated using timer frequency.. 5985.11 BogoMIPS (lpj=11970228)
\end{verbatim}
\begin{itemize}
\item Y acá el tiempo inicia.
\end{itemize}
\end{frame}

\begin{frame}[label={sec:orgd08ea6e},fragile]{Características de ejecución de procesos en sistema}
 \begin{center}
Límites y subsistemas de control
\end{center}
\begin{verbatim}
pid_max: default: 32768 minimum: 301
Security Framework initialized
AppArmor: AppArmor disabled by boot time parameter
Dentry cache hash table entries: 524288 (order: 10, 4194304 bytes)
Inode-cache hash table entries: 262144 (order: 9, 2097152 bytes)
Mount-cache hash table entries: 256
Initializing cgroup subsys cpuacct
Initializing cgroup subsys memory
Initializing cgroup subsys devices
Initializing cgroup subsys freezer
Initializing cgroup subsys net_cls
Initializing cgroup subsys blkio
Initializing cgroup subsys perf_event
\end{verbatim}
\end{frame}

\begin{frame}[label={sec:org59625fe},fragile]{Vamos subiendo de nivel: Dispositivos base}
 \begin{center}
Fundamental para multitarea: Cómo hablar con el temporizador, manejo
de interrupciones
\end{center}
\begin{verbatim}
..TIMER: vector=0x30 apic1=0 pin1=2 apic2=-1 pin2=-1
  (...)
NMI watchdog enabled, takes one hw-pmu counter.
Brought up 2 CPUs
Total of 2 processors activated (12011.18 BogoMIPS).
\end{verbatim}
\begin{center}
Y aquí (0.29s) comienzan a activarse los subsistemas que empleará el
usuario
\end{center}
\begin{verbatim}
devtmpfs: initialized
  (...)
NET: Registered protocol family 16
  (...)
[Firmware Bug]: ACPI: BIOS _OSI(Linux) query ignored
ACPI: Interpreter enabled
ACPI: (supports S0 S1 S3 S4 S5)
\end{verbatim}
\end{frame}

\begin{frame}[label={sec:org52a0d98},fragile]{Comienza enumeración de dispositivos}
 \begin{center}
Primer bus en ser \emph{barrido}: PCI. Proceso largo (0.46s–1.9s)
\end{center}
\begin{verbatim}
PCI: Using host bridge windows from ACPI; if necessary, use "pci=nocrs" and report a bug
ACPI: PCI Root Bridge [PCI0] (domain 0000 [bus 00-ff])
pci_root PNP0A03:00: host bridge window [io  0x0000-0x0cf7]
pci_root PNP0A03:00: host bridge window [io  0x0d00-0xffff]
pci_root PNP0A03:00: host bridge window [mem 0x000a0000-0x000bffff]
pci_root PNP0A03:00: host bridge window [mem 0x000c0000-0x000effff]
  (...)
pci 0000:00:00.0: [8086:2e10] type 0 class 0x000600
pci 0000:00:01.0: [8086:2e11] type 1 class 0x000604
pci 0000:00:01.0: PME# supported from D0 D3hot D3cold
  (...)
ACPI: PCI Interrupt Link [LNKA] (IRQs 3 4 5 6 7 9 10 *11 12 15)
ACPI: PCI Interrupt Link [LNKB] (IRQs 3 4 *5 6 7 9 10 11 12 15)
ACPI: PCI Interrupt Link [LNKC] (IRQs 3 4 5 6 7 *9 10 11 12 15)
\end{verbatim}
\end{frame}

\begin{frame}[label={sec:org29ed3d2},fragile]{Se reservan \emph{puertos} de memoria \(\rightarrow\) dispositivos}
 \begin{center}
Como veremos posteriormente, para \emph{hablar} con ciertos dispositivos
lo haremos a través de \emph{regiones de memoria} dedicadas a
entrada/salida
\end{center}
\begin{verbatim}
ACPI: bus type pnp registered
pnp 00:00: [bus 00-ff]
pnp 00:00: [io  0x0cf8-0x0cff]
pnp 00:00: [io  0x0000-0x0cf7 window]
pnp 00:00: [io  0x0d00-0xffff window]
pnp 00:00: [mem 0x000a0000-0x000bffff window]
pnp 00:00: [mem 0x000c0000-0x000effff window]
  (...)
pci 0000:00:1f.2: BAR 5: assigned [mem 0xf0000000-0xf00007ff]
pci 0000:00:03.0: BAR 0: assigned [mem 0xf0000800-0xf000080f 64bit]
pci 0000:00:1c.1: BAR 15: assigned [mem 0xf0100000-0xf02fffff 64bit pref]
\end{verbatim}
\end{frame}

\begin{frame}[label={sec:org59aaed3},fragile]{Encontramos conexiones con otros buses}
 \begin{center}
Vemos que el bus PCI es el \emph{maestro}, y de él descienden otros varios
buses
\end{center}
\begin{verbatim}
pci 0000:00:01.0: PCI bridge to [bus 01-01]
pci 0000:00:01.0:   bridge window [mem 0xfe500000-0xfe5fffff]
pci 0000:00:1c.0: PCI bridge to [bus 02-02]
pci 0000:00:1c.0:   bridge window [io  0x2000-0x2fff]
pci 0000:00:1c.0:   bridge window [mem 0xfe400000-0xfe4fffff]
pci 0000:00:1c.0:   bridge window [mem 0xf0300000-0xf04fffff 64bit pref]
pci 0000:00:1c.1: PCI bridge to [bus 03-03]
pci 0000:00:1c.1:   bridge window [io  0x1000-0x1fff]
pci 0000:00:1c.1:   bridge window [mem 0xfe300000-0xfe3fffff]
pci 0000:00:1c.1:   bridge window [mem 0xf0100000-0xf02fffff 64bit pref]
pci 0000:00:1e.0: PCI bridge to [bus 04-04]
pci 0000:00:01.0: setting latency timer to 64
pci 0000:00:1c.0: setting latency timer to 64
pci 0000:00:1c.1: setting latency timer to 64
pci 0000:00:1e.0: setting latency timer to 64
\end{verbatim}
\end{frame}

\begin{frame}[label={sec:orgf33c331},fragile]{Últimos toques antes de montar un micro-sistema}
 \begin{verbatim}
NET: Registered protocol family 2
IP route cache hash table entries: 131072 (order: 8, 1048576 bytes)
TCP established hash table entries: 524288 (order: 11, 8388608 bytes)
TCP bind hash table entries: 65536 (order: 8, 1048576 bytes)
TCP: Hash tables configured (established 524288 bind 65536)
TCP reno registered
UDP hash table entries: 2048 (order: 4, 65536 bytes)
UDP-Lite hash table entries: 2048 (order: 4, 65536 bytes)
NET: Registered protocol family 1
pci 0000:00:02.0: Boot video device
PCI: CLS 64 bytes, default 64
Unpacking initramfs...
Freeing initrd memory: 10904k freed
\end{verbatim}
\begin{itemize}
\item Con un \emph{ramdisk de inicio} montado termina la \emph{fase inicial} de la carga.
\item 1.98s \(\rightarrow\) 2.17s
\item El núcleo sigue reconociendo su entorno\ldots{}
\end{itemize}
\#+END\textsubscript{CENTER}
\end{frame}

\begin{frame}[label={sec:orgc9d6ad9},fragile]{Estructuras de gestión de memoria}
 \begin{itemize}
\item Veremos el rol de las \emph{tablas de traducción de direcciones}
(Translation Lookaside Buffer, TLB) en la unidad \emph{Administración de
memoria}
\end{itemize}
\begin{verbatim}
Placing 64MB software IO TLB between ffff8800c99fd000 - ffff8800cd9fd000
software IO TLB at phys 0xc99fd000 - 0xcd9fd000
Simple Boot Flag at 0x7a set to 0x1
audit: initializing netlink socket (disabled)
type=2000 audit(1362187934.168:1): initialized
HugeTLB registered 2 MB page size, pre-allocated 0 pages
\end{verbatim}
\end{frame}

\begin{frame}[label={sec:org6dd650c},fragile]{Comienzan las cargas de módulos controladores}
 \begin{itemize}
\item Llegamos aquí a los 2.18s
\item Comienza la parte más lenta del inicio
\begin{itemize}
\item La inicialización de cada dispositivo tarda en promedio entre
\(\frac{1}{100}s\) y \(\frac{1}{10}s\)
\end{itemize}
\end{itemize}
\begin{verbatim}
VFS: Disk quotas dquot_6.5.2
Block layer SCSI generic (bsg) driver version 0.4 loaded (major 253)
io scheduler noop registered
io scheduler deadline registered
io scheduler cfq registered (default)
  (...)
vesafb: mode is 1024x768x16, linelength=2048, pages=0
vesafb: framebuffer at 0xd0000000, mapped to 0xffffc90011100000, using 1536k, total 1536k
Console: switching to colour frame buffer device 128x48
  (...)
Serial: 8250/16550 driver, 4 ports, IRQ sharing enabled
serial8250: ttyS0 at I/O 0x3f8 (irq = 4) is a 16550A
00:07: ttyS0 at I/O 0x3f8 (irq = 4) is a 16550A
\end{verbatim}
\end{frame}

\begin{frame}[label={sec:org120af62},fragile]{\emph{Me parece} que ya hay hilos dentro del núcleo}
 \begin{itemize}
\item Vemos que se van inicializando elementos con poca relación entre sí
\item Indicio de hilos que van avanzando en paralelo
\end{itemize}
\begin{verbatim}
mousedev: PS/2 mouse device common for all mice
rtc_cmos 00:05: RTC can wake from S4
rtc_cmos 00:05: rtc core: registered rtc_cmos as rtc0
rtc0: alarms up to one day, 242 bytes nvram, hpet irqs
cpuidle: using governor ladder
cpuidle: using governor menu
TCP cubic registered
NET: Registered protocol family 10
Mobile IPv6
NET: Registered protocol family 17
Registering the dns_resolver key type
  (...)
rtc_cmos 00:05: setting system clock to 2013-03-02 01:32:15 UTC (1362187935)
Freeing unused kernel memory: 576k freed
Write protecting the kernel read-only data: 6144k
\end{verbatim}
\end{frame}

\begin{frame}[label={sec:orgcf1de37},fragile]{Más subsistemas: UDev, USBcore, EHCI, ATA\ldots{}}
 \begin{verbatim}
udevd[52]: starting version 175
  (...)
usbcore: registered new interface driver usbfs
usbcore: registered new interface driver hub
usbcore: registered new device driver usb
ehci_hcd: USB 2.0 'Enhanced' Host Controller (EHCI) Driver
libata version 3.00 loaded.
uhci_hcd: USB Universal Host Controller Interface driver
  (...)
usb usb1: New USB device found, idVendor=1d6b, idProduct=0002
usb usb1: New USB device strings: Mfr=3, Product=2, SerialNumber=1
usb usb1: Product: EHCI Host Controller
usb usb1: Manufacturer: Linux 3.2.0-4-amd64 ehci_hcd
usb usb1: SerialNumber: 0000:00:1a.7
hub 1-0:1.0: USB hub found
hub 1-0:1.0: 6 ports detected
ata_generic 0000:00:03.2: setting latency timer to 64
scsi0 : ata_generic
scsi1 : ata_generic
ata1: PATA max UDMA/100 cmd 0xfe80 ctl 0xfe90 bmdma 0xfef0 irq 18
\end{verbatim}
\end{frame}

\begin{frame}[label={sec:org876fbcd},fragile]{\emph{Caminando} los buses}
 \begin{itemize}
\item A partir de este punto, el núcleo averigua lo que tiene conectado
por el bus USB
\item Líneas bastante repetitivas
\item Desde los 2.64 hasta los 3.0 segundos
\item Comienzan a aparecer los discos SATA/SCSI (por \texttt{libata})
\end{itemize}
\end{frame}

\begin{frame}[label={sec:org2fe1327},fragile]{\emph{Caminando} los buses: SATA}
 \begin{verbatim}
ata8: SATA link down (SStatus 0 SControl 300)
ata6: SATA link up 1.5 Gbps (SStatus 113 SControl 300)
ata4: SATA link down (SStatus 0 SControl 300)
ata5: SATA link up 1.5 Gbps (SStatus 113 SControl 300)
ata3: SATA link up 3.0 Gbps (SStatus 123 SControl 300)
  (...)
scsi 2:0:0:0: Direct-Access     ATA      WDC WD5000AAKS-2 12.0 PQ: 0 ANSI: 5
scsi 4:0:0:0: CD-ROM            TSSTcorp DVD+-RW TS-H653G DW10 PQ: 0 ANSI: 5
sd 2:0:0:0: [sda] 976773168 512-byte logical blocks: (500 GB/465 GiB)
sd 2:0:0:0: [sda] Write Protect is off
sd 2:0:0:0: [sda] Mode Sense: 00 3a 00 00
sd 2:0:0:0: [sda] Write cache: enabled, read cache: enabled, doesn't support DPO or FUA
sr0: scsi3-mmc drive: 48x/48x writer dvd-ram cd/rw xa/form2 cdda tray
\end{verbatim}
\end{frame}

\begin{frame}[label={sec:orgd5812f2},fragile]{Por fin: Sistemas de archivos}
 \begin{itemize}
\item Al cargar los controladores de sistemas de archivos (3.55s), ya
podemos montar y comenzar a iniciar el \emph{entorno operativo}
\item Esto sigue \emph{sumergido} entre mensajes de descubrimiento de
dispositivos (principalmente unidades de disco y dispositivos USB)
\item Veremos posteriormente características de ReiserFS y de otros
sistemas de archivos
\end{itemize}
\begin{verbatim}
REISERFS (device dm-0): found reiserfs format "3.6" with standard journal
REISERFS (device dm-0): using ordered data mode
reiserfs: using flush barriers
REISERFS (device dm-0): journal params: device dm-0, size 8192, journal first block 18, max trans len 1024, max batch 900, max commit age 30, max trans age 30
REISERFS (device dm-0): checking transaction log (dm-0)
REISERFS (device dm-0): Using r5 hash to sort names
\end{verbatim}
\end{frame}

\begin{frame}[label={sec:org7614bc8},fragile]{Algunos puntos sueltos\ldots{}}
 \begin{itemize}
\item La inicialización continúa
\item En este caso, \emph{caminar} los buses USB llevó hasta los 5.06s
\item \emph{Silencio} de 11.4s después de que inicia \texttt{udevd[347]}
\begin{itemize}
\item Puede estar estructurando lo encontrado en el sistema \texttt{/dev}
dinámico (?)
\end{itemize}
\item Otras líneas interesantes (sueltas):
\end{itemize}
\begin{verbatim}
input: PC Speaker as /devices/platform/pcspkr/input/input0
input: Power Button as /devices/LNXSYSTM:00/device:00/PNP0C0C:00/input/input1
Adding 2097148k swap on /dev/mapper/mosca-swap.  Priority:-1 extents:1 across:2097148k
EXT4-fs (sda2): mounted filesystem with ordered data mode. Opts: (null)
kjournald starting.  Commit interval 5 seconds
\end{verbatim}
\end{frame}

\section{Inicio del espacio de usuario}
\label{sec:org2238a57}
\begin{frame}[label={sec:orgd48d11c},fragile]{¡Bienvenido a tu sistema!}
 \begin{itemize}
\item Tras este punto, estamos inequívocamente ya corriendo en un sistema
completo
\item La primer evidencia de ello: La carga de \texttt{udevd} como proceso >
\texttt{pid\_min} (301) a los 5.06s
\item Ahora, ¿qué sigue?
\end{itemize}
\end{frame}

\begin{frame}[label={sec:orgf2e241a},fragile]{El primer proceso: \texttt{init}}
 \begin{itemize}
\item En un sistema Unix, tan pronto el núcleo está listo para ejecutar
\emph{algo} ejecuta al proceso \texttt{/sbin/init}
\item \texttt{init} se encarga de inicializar al resto del sistema
\begin{itemize}
\item ¿Cómo?
\end{itemize}
\item Estrategias \emph{clásicas}: Sistemas BSD, SysV
\begin{itemize}
\item Todos los sistemas tipo Unix, hasta hace unos cinco años siguen
alguna de estas dos vías
\end{itemize}
\item Estrategias \emph{modernas}:
\begin{itemize}
\item Orientado a \emph{eventos}: \texttt{upstart}
\item Orientado a \emph{sockets}: \texttt{systemd}
\end{itemize}
\end{itemize}
\end{frame}

\begin{frame}[label={sec:org0aed699},fragile]{Esquema BSD (1)}
 \begin{center}
\texttt{init} sigue una serie de simples \emph{scripts} para sus principales
operaciones (de: \href{http://openbsd.org/faq/es/faq10.html}{Gestión del sistema OpenBSD}):
\end{center}
\begin{description}
\item[{/etc/rc}] Fichero de configuración principal. No se debe editar.
\item[{/etc/rc.conf}] Fichero de configuración usado por /etc/rc para
saber qué dæmon deben iniciarse con el sistema.
\item[{/etc/rc.conf.local}] Fichero de configuración que se puede usar
para anular configuraciones de /etc/rc.conf, sin que sea
necesario tocar el fichero /etc/rc.conf; muy conveniente para
los usuarios que actualizan el sistema con frecuencia.
\end{description}
(continúa)
\end{frame}

\begin{frame}[label={sec:org737c7a8}]{Esquema BSD (2)}
\begin{description}
\item[{/etc/netstart}] Fichero de configuración usado para iniciar la red. No se debe editar.
\item[{/etc/rc.local}] Fichero de configuración usado para tareas de
administración local. Aquí se debe almacenar
información específica de la máquina anfitriona o
de los dæmon.
\item[{/etc/rc.securelevel}] Fichero de configuración para ejecutar las
órdenes que deben ser invocadas antes de los cambios en el
nivel de seguridad. Véase init(8).
\item[{/etc/rc.shutdown}] Fichero de configuración invocado por
shutdown(8). En este fichero se debe añadir cualquier cosa que
se quiera hacer antes de cerrar el sistema. Véase
rc.shutdown(8).
\end{description}
\end{frame}

\begin{frame}[label={sec:orgd9bfedc},fragile]{Esquema SysV (1)}
 \begin{itemize}
\item Parte de la idea de diferentes \emph{conjuntos de programas} que pueden
ser requeridos por distintos \emph{usos del sistema} (\emph{runlevels})
\item Busca menor dependencia en el administrador para indicar qué se va
a iniciar
\begin{itemize}
\item Permitiendo que los distintos \emph{paquetes} se ubiquen en el punto correcto
\end{itemize}
\item Configuración basada en un archivo base, \texttt{/etc/inittab}, y un
directorio por runlevel (\texttt{/etc/rc0.d/} a \texttt{/etc/rc6.d/})
\end{itemize}
\end{frame}

\begin{frame}[label={sec:orgb60e50f},fragile]{Esquema SysV (2)}
 \begin{itemize}
\item En cada directorio se colocan archivos (ejecutados en órden
\emph{alfabético}) indicando si \emph{inician} (\texttt{S}) o \emph{detienen} (\texttt{K}) al
servicio
\begin{itemize}
\item Esta convención apunta a que p.ej. \texttt{/etc/rc3.d/S20gdm3} inicia
\texttt{gdm3} al estar en runlevel 3, y con un \emph{ordenamiento} de 20
\end{itemize}
\item Los runlevels \emph{no son acumulativos} (no hay relación que un nivel
\emph{contenga} al anterior)
\item Algunos \emph{runlevels} tienen significado especial
\begin{description}
\item[{0}] Shutdown (apagado)
\item[{1}] Mantenimiento / monousuario
\item[{6}] Restart (reinicio)
\end{description}
\end{itemize}
\end{frame}
\end{document}