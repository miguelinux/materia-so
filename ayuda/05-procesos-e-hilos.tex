% Created 2023-02-06 Mon 12:04
% Intended LaTeX compiler: pdflatex
\documentclass[presentation]{beamer}
\usepackage[utf8]{inputenc}
\usepackage[T1]{fontenc}
\usepackage{graphicx}
\usepackage{longtable}
\usepackage{wrapfig}
\usepackage{rotating}
\usepackage[normalem]{ulem}
\usepackage{amsmath}
\usepackage{amssymb}
\usepackage{capt-of}
\usepackage{hyperref}
\usepackage[spanish]{babel}
\usepackage{listings}
\pgfdeclareimage[height=0.7\textheight]{../img/pres/cintillo.png}{../img/pres/cintillo.png}\logo{\pgfuseimage{../img/pres/cintillo.png}}
\AtBeginSection[]{ \begin{frame}<beamer> \frametitle{Índice} \tableofcontents[currentsection] \end{frame} }
\definecolor{string}{rgb}{0,0.6,0} \definecolor{shadow}{rgb}{0.5,0.5,0.5} \definecolor{keyword}{rgb}{0.58,0,0.82} \definecolor{identifier}{rgb}{0,0,0.7}
\renewcommand{\ttdefault}{pcr}
\lstset{basicstyle=\ttfamily\scriptsize\bfseries, showstringspaces=false, keywordstyle=\color{keyword}, stringstyle=\color{string}, identifierstyle=\color{identifier}, commentstyle=\mdseries\textit, inputencoding=utf8, extendedchars=true, breaklines=true, breakatwhitespace=true, breakautoindent=true, numbers=left, numberstyle=\ttfamily\tiny\textit}
\newcommand{\rarrow}{$\rightarrow$\hskip 0.5em}
\usetheme{Warsaw}
\usecolortheme{lily}
\author{Gunnar Wolf}
\date{}
\title{Administración de procesos: Procesos e hilos}
\hypersetup{
 pdfauthor={Gunnar Wolf},
 pdftitle={Administración de procesos: Procesos e hilos},
 pdfkeywords={},
 pdfsubject={},
 pdfcreator={Emacs 28.2 (Org mode 9.5.5)}, 
 pdflang={Spanish}}
\begin{document}

\maketitle

\section{Concepto y estados de un proceso}
\label{sec:org8674b1c}

\begin{frame}[label={sec:orga0070e7}]{¿Cuándo es proceso? ¿Cuando es programa?}
\begin{description}
\item[{Programa}] Una lista de instrucciones a seguir, una \emph{entidad
pasiva}
\item[{Proceso}] \emph{Entidad activa} que:
\begin{itemize}
\item \emph{Emplea} al programa
\item Típicamente opera sobre un \emph{conjunto de datos}
\item Tiene \emph{información de estado} que indica, entre otras cosas, en
qué punto va la ejecución.
\end{itemize}
\end{description}
\end{frame}

\begin{frame}[label={sec:org3f16589}]{¿Cuando es proceso? ¿Cuándo es tarea?}
\begin{description}
\item[{Tarea}] Equivalente a un proceso en un \emph{sistema por lotes};
requiere típicamente menos metainformación.
\end{description}
\begin{center}
La distinción proceso-tarea no es del todo clara u objetiva.
\vfill
Hay textos que emplean uno u otro término indistintamente
\vfill
Nosotros emplearemos siempre el término \emph{proceso}.
\end{center}
\end{frame}

\begin{frame}[label={sec:orgb298802}]{La ilusión de la concurrencia}
\begin{itemize}
\item Un sistema actual nos da la \emph{ilusión} de ejecución simultánea de
muchos procesos
\item La realidad: Casi todos están suspendidos, esperando que los active
el planificador
\item En un momento dado sólo puede estarse ejecutando un número de
procesos igual o menor al número de CPUs que tenga el sistema.
\item Esa ilusión tiene grandes costos\ldots{} Especialmente pensando con
suficiente velocidad
\end{itemize}
\end{frame}

\section{El proceso}
\label{sec:orgc3f2d36}

\begin{frame}[label={sec:orge00af58}]{Estados de un proceso}
Un proceso, a lo largo de su vida, alterna entre diferentes \emph{estados}
de ejecución. Estos son:

\begin{description}
\item[{Nuevo}] Se solicitó al sistema la creación de un proceso, y sus
recursos y estructuras están siendo creadas

\item[{Listo}] Está listo para ser asignado para su ejecución

\item[{En ejecución}] El proceso está siendo ejecutado

\item[{Bloqueado}] En espera de algún evento para poder continuar
ejecutándose

\item[{Terminado}] El proceso terminó de ejecutarse; sus estructuras están
a la espera de ser \emph{limpiadas} por el sistema
\end{description}
\end{frame}

\begin{frame}[label={sec:org899c864}]{Estados de un proceso}
\begin{figure}[htbp]
\centering
\includegraphics[height=0.7\textheight]{../img/dot/estados_proceso.png}
\caption{Diagrama de transición entre los estados de un proceso}
\end{figure}
\end{frame}

\begin{frame}[label={sec:orgc48f019}]{El bloque de control del proceso (PCB) (1)}
\begin{center}
¿Qué información debe mantener el sistema acerca de cada proceso?

O, ¿Cuánto cuesta un cambio de contexto?
\end{center}

\begin{description}
\item[{Estado del proceso}] El estado actual del proceso

\item[{Contador de programa}] Cuál es la siguiente instrucción a ser
ejecutada

\item[{Registros del CPU}] Información específica del estado del CPU

\item[{Información de planificación (scheduling)}] La prioridad del
proceso, la \emph{cola} en que está agendado, y demás información que
puede ayudar al sistema operativo a agendar al proceso
\end{description}
\end{frame}

\begin{frame}[label={sec:org39082bf}]{El bloque de control del proceso (PCB) (2)}
\begin{description}
\item[{Información de administración de memoria}] Las tablas de mapeo de
memoria (páginas o segmentos, dependiendo del sistema
operativo).

\item[{Estado de E/S}] Listado de dispositivos y archivos asignados que el
proceso tiene \emph{abiertos} en un momento dado.

\item[{Información de contabilidad}] Información de la utilización de
recursos que ha tenido este proceso
\begin{itemize}
\item Tiempo \emph{de usuario} y \emph{de sistema}
\item Uso acumulado de memoria y dispositivos
\item etc.
\end{itemize}
\end{description}
\end{frame}

\section{Hilos}
\label{sec:orgc621d9b}

\begin{frame}[label={sec:org8ef5a5e}]{El problema con los hilos}
\begin{quote}
Aunque los hilos parecen ser un pequeño paso partiendo del cómputo
secuencial, de hecho, son un paso inmenso. Descartan las propiedades
más esenciales y atractivas del cómputo secuencial: Facilidad de
comprensión, predictabilidad y determinismo. Los hilos, como un modelo
de computación, son salvajemente no-determinísticos, y el trabajo del
programador se convierte en \emph{podar} ese no-determinismo.

— \emph{El problema con los hilos}, Edward A. Lee, UC Berkeley, 2006
\end{quote}
\end{frame}

\begin{frame}[label={sec:orge07f904}]{El peso de los procesos}
\begin{itemize}
\item La cantidad de información implicada en un cambio de contexto es
muy grande
\item Desperdicio \emph{burocrático} de recursos
\item Una respuesta: \emph{procesos ligeros} (\emph{Lightweight processes}, LWP)
\begin{itemize}
\item Diversos \emph{hilos de ejecución} dentro de un mismo proceso
\end{itemize}
\end{itemize}
\end{frame}

\begin{frame}[label={sec:orgb537ccf}]{Extrapolando}
\begin{itemize}
\item Un proceso que no usa hilos es \emph{un sólo hilo}
\item Un sistema operativo que no ofrece soporte para hilos agendaría a
nuestro proceso como a cualquier otro
\end{itemize}
\end{frame}

\begin{frame}[label={sec:org7ce10d2}]{La diferencia desde dentro}
\begin{itemize}
\item Entre diferentes procesos, cada uno tiene ilusión de \emph{exclusividad
virtual} sobre la computadora
\begin{itemize}
\item Espacio de direccionamiento de memoria exclusivo
\item Descriptores de archivos y dispositivos independientes
\end{itemize}
\item Los hilos \emph{son y deben ser conscientes} de su coexistencia
\begin{itemize}
\item Comparten memoria, descriptores de archivos y dispositivos
\item Pueden tener \emph{variables} locales y globales
\end{itemize}
\end{itemize}
\end{frame}

\begin{frame}[label={sec:org26ee20b}]{¿Qué tanto se \emph{aligera} el PCB?}
\begin{itemize}
\item Estado del proceso
\item Cada hilo se ejecuta de forma aparentemente secuencial
\begin{itemize}
\item Tiene su propio \emph{contador de programa}
\item Y su propia pila de llamadas (\emph{stack})
\end{itemize}
\item Conjunto de variables locales
\begin{itemize}
\item Puede ser implementado de forma muy sencilla
\item p.ej. con arreglos indexados por ID de hilo
\end{itemize}
\item Información de planeación \emph{interna}
\item Pueden llevar información de contabilidad
\item No requerimos:
\begin{itemize}
\item Registros del CPU
\item Estado de E/S
\end{itemize}
\end{itemize}
\end{frame}

\begin{frame}[label={sec:org9a0f4ab}]{Patrones de trabajo: Jefe / trabajador}
\begin{itemize}
\item Un hilo tiene una tarea distinta de todos los demás
\item El hilo \emph{jefe} genera o recopila tareas
\item Los hilos \emph{trabajadores} efectúan el trabajo.
\item Modelo común para procesos servidor, GUIs que procesan eventos
\item El jefe puede llevar la contabilidad de los trabajos realizados
\end{itemize}
\end{frame}
\begin{frame}[label={sec:orga644299}]{Patrones de trabajo: Jefe / trabajador}
\begin{figure}[htbp]
\centering
\includegraphics[height=0.7\textheight]{../img/dot/jefe_trabajador.png}
\caption{Patrón de hilos jefe/trabajador}
\end{figure}
\end{frame}

\begin{frame}[label={sec:orgf0e7675}]{Patrones de trabajo: Equipo de trabajo}
\begin{itemize}
\item A partir de hilos idénticos
\item Realizarán las mismas tareas sobre diferentes datos
\item Muy frecuentemente utilizado para cálculos matemáticos
(p.ej. criptografía, render).
\item Puede combinarse con jefe/trabajador para generar
previsualizaciones (la tarea se realiza progresivamente)
\end{itemize}
\end{frame}

\begin{frame}[label={sec:org9b110c3}]{Patrones de trabajo: Equipo de trabajo}
\begin{figure}[htbp]
\centering
\includegraphics[height=0.7\textheight]{../img/dot/equipo_trab.png}
\caption{Patrón de hilos \emph{Equipo de trabajo}}
\end{figure}
\end{frame}

\begin{frame}[label={sec:orgfd7ad1e}]{Patrones de trabajo: Línea de ensamblado}
\begin{itemize}
\item Una tarea larga que puede dividirse en pasos
\item Cada hilo se enfoca en una sóla tarea
\item Pasa los datos a otro hilo conforme va terminando
\item Ayuda a mantener las rutinas simples de comprender
\item Permite continuar procesando datos cuando hay hilos esperando E/S
\end{itemize}
\end{frame}

\begin{frame}[label={sec:orge924a51}]{Patrones de trabajo: Línea de ensamblado}
\begin{figure}[htbp]
\centering
\includegraphics[width=\textwidth]{../img/dot/linea_ensamblado.png}
\caption{Patrón de hilos \emph{Línea de ensamblado}}
\end{figure}
\end{frame}

\section{Los hilos para el sistema operativo}
\label{sec:orge13ee33}

\begin{frame}[label={sec:orgbca0312}]{Hilos de usuario (o hilos verdes)}
\begin{itemize}
\item Los hilos pueden implementarse 100\% con las facilidades del proceso
\begin{itemize}
\item Caso extremo: Programas multihilos en sistemas operativos mínimos
(o directo sobre el hardware)
\end{itemize}
\item Mayor portabilidad
\item A través de alguna biblioteca \emph{del lenguaje/entorno de programación}
\item Típicamente multitarea \emph{interna} cooperativa
\end{itemize}
\end{frame}

\begin{frame}[label={sec:orgf214793}]{Pros y contras de los hilos de usuario}
\begin{center}
Ganamos:
\end{center}
\begin{itemize}
\item Espacio de memoria compartido sin intervención del sistema operativo
\item Mayor rapidez para realizar trabajos cooperativos o multiagentes
\end{itemize}
\begin{center}
Perdemos:
\end{center}
\begin{itemize}
\item Cualquier llamada al sistema interrumpe a \emph{todos los hilos}
\item No aprovechan el multiprocesamiento
\end{itemize}
\end{frame}

\begin{frame}[label={sec:org434ed52}]{Hilos nativos (o hilos de kernel)}
\begin{itemize}
\item A través de bibliotecas de sistema que \emph{informan} al sistema
\item El núcleo se encarga de la multitarea \emph{preventiva} entre los hilos
\item El proceso \emph{puede} pedir al sistema un mayor número de procesadores
\begin{itemize}
\item Logrando ejecución verdaderamente paralela
\end{itemize}
\item El sistema puede priorizar de diferente manera a un proceso
multihilo
\end{itemize}
\end{frame}

\section{Concurrencia}
\label{sec:org3ef014b}

\begin{frame}[label={sec:org840bb7d}]{Referencia bibliográfica}
\begin{center}
Para este tema, recomiendo fuertemente revisen «\href{http://greenteapress.com/semaphores/LittleBookOfSemaphores.pdf}{The little book of
semaphores}» de Allen Downey (2008) (disponible para su descarga como
PDF).
\end{center}
\end{frame}

\begin{frame}[label={sec:orge1ed841}]{¿Qué es concurrencia?}
\begin{itemize}
\item Quedó ya claro que (para nosotros) concurrencia no significa que dos
o más eventos ocurran al mismo tiempo
\item Nos referimos a dos o más eventos \emph{cuyo órden no es determinista}
\item No podemos predecir el órden relativo en que ocurrirán
\end{itemize}
\end{frame}

\begin{frame}[label={sec:orgbc050d8}]{¿Cuándo hablamos de concurrencia?}
\begin{itemize}
\item Dos o más hilos del mismo proceso
\item Dos o más procesos en la misma computadora
\item Dos o más procesos en computadoras separadas conectadas por red
\item Dos o más procesos que ocurran sin conexión alguna y posteriormente
requieran sincronización
\end{itemize}
\end{frame}

\begin{frame}[label={sec:org5ddc493}]{La sincronización implica involucramiento del programador}
\begin{center}
El sistema operativo brinda la \emph{ilusión} a cada proceso de estar
ejecutando en una computadora dedicada, pero\ldots{}
\end{center}
\begin{itemize}
\item El proceso puede depender de datos obtenidos en fuentes externas
\item Puede ser necesario esperar a que otro proceso haya pasado cierto
punto en su ejecución
\begin{itemize}
\item …Que tengamos ya los resultados parciales de un cómputo paralelo
\item …O que no más de \emph{m} o menos de \emph{n} procesos estén en determinado punto
\item …O notificar a otro proceso de nuestro avance, o…
\end{itemize}
\end{itemize}
\end{frame}

\begin{frame}[label={sec:org5a85581}]{La sincronización como comunicación entre procesos (IPC)}
\begin{itemize}
\item En esta sección veremos ejemplos de \emph{primitivas de sincronización}
\item Todas ellas son modalidades de \emph{comunicación entre procesos} (IPC)
\item Nos centraremos principalmente en los \emph{semáforos}
\end{itemize}
\end{frame}

\begin{frame}[label={sec:org935b40e}]{Problemas clásicos para ir pensando}
\begin{itemize}
\item Problema productor-consumidor
\item Problema lectores-escritores
\item La cena de los filósofos
\item Los fumadores compulsivos
\end{itemize}

\begin{center}
…Y hay \(\approx\) 20 problemas más explicados en \emph{The little book of
semaphores}.
\vfill
¡Revísenlo!
\end{center}
\end{frame}

\begin{frame}[label={sec:orgc9c0cc0}]{Problema productor-consumidor}
\begin{itemize}
\item Pensemos un entorno multihilo como una línea de ensamblado
\begin{itemize}
\item Algunos hilos \emph{producen} ciertas estructuras (p.ej. eventos en un
GUI)
\item Otros hilos las \emph{consumen} (procesan los eventos)
\end{itemize}
\item Los eventos se \emph{apilan} en un buffer disponible para todos los hilos
\item ¿Cómo podemos asegurar que dos hilos no modifiquen al buffer al
mismo tiempo?
\item ¿Cómo podemos evitar que los consumidores hagan \emph{espera activa}?
\end{itemize}
\end{frame}

\begin{frame}[label={sec:orgc43a85a}]{Problema lectores-escritores}
\begin{itemize}
\item Muchos procesos \emph{lectores} pueden usar simultáneamente una estrucura
\begin{itemize}
\item Si algún proceso \emph{escritor} la requiere, debemos evitar que
cualquier \emph{lector} esté activo
\end{itemize}
\item Requisitos de sincronización:
\begin{itemize}
\item Sin límite en la cantidad de \emph{lectores activos}
\item Los escritores deben tener \emph{acceso exclusivo} a la \emph{sección crítica}
\item Evitar \emph{inanición} de escritores por exceso de lectores
\end{itemize}
\end{itemize}
\end{frame}

\begin{frame}[label={sec:org800e85a}]{La cena de los filósofos (Edsger Dijkstra, 1965)}
\begin{itemize}
\item Una mesa redonda
\item Tazón de arroz al centro
\item Cinco platos, cinco filósofos, cinco palillos chinos
\item Los filósofos piensan hasta tener hambre.
\begin{itemize}
\item Cuando tienen hambre, buscan comer
\item No hablan entre sí
\end{itemize}
\item Sólo un filósofo puede sostener un palillo a la vez
\item ¿Qué puede salir mal?
\begin{itemize}
\item ¿Cómo evitarlo con sólo primitivas de sincronización?
\end{itemize}
\end{itemize}
\end{frame}

\begin{frame}[label={sec:org935603f}]{Los fumadores compulsivos (Suhas Patil, 1971)}
\begin{itemize}
\item Tres fumadores empedernidos
\begin{itemize}
\item Con cantidades ilimitadas de uno de tres ingredientes cada uno:
Tabaco, papel, cerillos
\end{itemize}
\item Un \emph{agente} que consigue ingredientes independientemente
\item ¿Cómo asegurarse de que los recursos se utilizan siempre, \emph{tan
pronto como estén disponibles}?
\begin{itemize}
\item Sin que el agente sepa quién tiene qué ingrediente
\end{itemize}
\end{itemize}
\end{frame}
\end{document}