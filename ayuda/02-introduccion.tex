% Created 2023-02-06 Mon 12:03
% Intended LaTeX compiler: pdflatex
\documentclass[presentation]{beamer}
\usepackage[utf8]{inputenc}
\usepackage[T1]{fontenc}
\usepackage{graphicx}
\usepackage{longtable}
\usepackage{wrapfig}
\usepackage{rotating}
\usepackage[normalem]{ulem}
\usepackage{amsmath}
\usepackage{amssymb}
\usepackage{capt-of}
\usepackage{hyperref}
\usepackage[spanish]{babel}
\usepackage{listings}
\pgfdeclareimage[height=0.7\textheight]{../img/pres/cintillo.png}{../img/pres/cintillo.png}\logo{\pgfuseimage{../img/pres/cintillo.png}}
\AtBeginSection[]{ \begin{frame}<beamer> \frametitle{Índice} \tableofcontents[currentsection] \end{frame} }
\definecolor{string}{rgb}{0,0.6,0} \definecolor{shadow}{rgb}{0.5,0.5,0.5} \definecolor{keyword}{rgb}{0.58,0,0.82} \definecolor{identifier}{rgb}{0,0,0.7}
\renewcommand{\ttdefault}{pcr}
\lstset{basicstyle=\ttfamily\scriptsize\bfseries, showstringspaces=false, keywordstyle=\color{keyword}, stringstyle=\color{string}, identifierstyle=\color{identifier}, commentstyle=\mdseries\textit, inputencoding=utf8, extendedchars=true, breaklines=true, breakatwhitespace=true, breakautoindent=true, numbers=left, numberstyle=\ttfamily\tiny\textit}
\newcommand{\rarrow}{$\rightarrow$\hskip 0.5em}
\usetheme{Warsaw}
\usecolortheme{lily}
\author{Gunnar Wolf}
\date{}
\title{Introducción a los sistemas operativos}
\hypersetup{
 pdfauthor={Gunnar Wolf},
 pdftitle={Introducción a los sistemas operativos},
 pdfkeywords={},
 pdfsubject={},
 pdfcreator={Emacs 28.2 (Org mode 9.5.5)}, 
 pdflang={Spanish}}
\begin{document}

\maketitle

\section{¿Qué son y qué hacen?}
\label{sec:orgf37f9c8}

\begin{frame}[label={sec:orgf0abfe9}]{¿Qué es un sistema operativo?}
\begin{itemize}
\item El \emph{sistema base} de una computadora \pause
\item El programa que \emph{siempre corre} \pause
\item Gestor de los \emph{recursos} del sistema \pause
\item Lo que define \emph{qué es compatible} y \emph{qué no} dentro de una
determinada arquitectura \pause
\item El programa \emph{menos importante} de todos \pause
\begin{itemize}
\item No realiza \emph{trabajo útil}, sino que permite que \emph{otros} lo hagan
\end{itemize}
\item …
\end{itemize}
\end{frame}

\begin{frame}[label={sec:orgbdc3bdb}]{…¿Qué \emph{no} es?}
\begin{itemize}
\item Los programas básicos para administrar archivos
\item La forma en que el usuario \emph{lanza} programas
\item El ambiente con que interactúa el usuario
\begin{itemize}
\item Entorno gráfico
\item Línea de comando
\item …
\end{itemize}
\end{itemize}
\end{frame}

\begin{frame}[label={sec:org0bf9e3e}]{¿Qué brinda al \emph{programador}?}
\begin{itemize}
\item Abstracción
\begin{itemize}
\item \emph{Embellecimiento} -Finkel
\item \emph{Virtualización} -Arpaci-Dusseau
\end{itemize}
\item Gestión de recursos
\item Aislamiento y protección
\end{itemize}
\end{frame}
\begin{frame}[label={sec:org7f21f41}]{Abstracción}
\begin{center}
Proporciona \emph{abstracciones consistentes} y \emph{simplificaciones} a los
procesos del usuario
\end{center}
\begin{itemize}
\item Archivos y directorios
\item \emph{Flujos de caracteres} (entrada/salida)
\item Dispositivos, conexiones de red, contacto con el \emph{mundo exterior}
\item El concepto mismo de \emph{proceso}
\end{itemize}
\end{frame}

\begin{frame}[label={sec:orgdc8b8f5}]{Gestión de recursos}
\begin{center}
Administra los \emph{recursos existentes} en la computadora, permitiendo la
ejecución a los diversos procesos
\end{center}
\begin{itemize}
\item Cómo comparten los diversos procesos los recursos existentes (y
rivales)
\item Políticas de asignación (y recuperación) justas
\end{itemize}
\end{frame}

\begin{frame}[label={sec:orgcc46968}]{Aislamiento y protección}
\begin{center}
Protección de los datos, de los recursos, de los procesos
\end{center}
\begin{itemize}
\item Entre procesos
\item Entre usuarios
\item Ante procesos \emph{mal comportados}
\item Ante procesos maliciosos
\end{itemize}
\end{frame}

\begin{frame}[label={sec:orgdeec494}]{Ahora sí… ¿Qué es un sistema operativo?}
\begin{itemize}
\item El principal programa que corre en una computadora de propósito
general
\item Provee una serie de abstracciones básicas a los programas
\begin{itemize}
\item Pueden haber diferentes sistemas operativos, definiendo distintas
\emph{interfaces}, sobre el mismo hardware
\item Un mismo sistema operativo puede \emph{adecuarse a} distintas
arquitecturas de hardware
\end{itemize}
\item Ofrece una infraestructura de gestión, aislamiento y protección de
recursos
\item Permite la implementación de \emph{entornos operativos}
\end{itemize}
\end{frame}

\section{Historia y evolución}
\label{sec:org4a27faf}

\begin{frame}[label={sec:org219a62d}]{Construyendo a través de la historia}
\begin{center}
Para comprender lo que hoy gestionan los sistemas operativos,
comencemos viendo cómo es que llegaron a gestionarlo.
\vfill
Vamos con un repaso histórico de la historia de la computación,
enfocados a las \emph{inovaciones} de cada etapa
\end{center}
\end{frame}

\begin{frame}[label={sec:org2dee163}]{En el principio…}
\begin{itemize}
\item Arquitectura \emph{von Neumann} (programa almacenado)
\item Programación directa y explícita para el hardware
\item Tiempo de programación \(\rightarrow\) tiempo \emph{no productivo}
\(\rightarrow\) desperdicio de recursos
\end{itemize}
\end{frame}

\begin{frame}[label={sec:org3b0901a}]{Sistemas de proceso por lotes (\emph{batch processing})}
\begin{itemize}
\item Los programadores codifican su código en un medio de almacenamiento
(\emph{tarjetas perforadas})
\item Entregan los tambaches (\emph{batches}) a los operadores
\item Los operadores cargan secuencialmente los trabajos, entregan los
resultados conforme se presentan
\end{itemize}
\end{frame}

\begin{frame}[label={sec:org1566b45}]{Sistemas monitor en el proceso por lotes}
\begin{itemize}
\item Implementan protección evitando la corrupción de \emph{otros trabajos}
\begin{itemize}
\item Interactuar con el lector de tarjetas (corrompiendo el siguiente
programa)
\item Temporizadores y alarmas para evitar ciclos infinitos
\end{itemize}
\item Estas medidas de protección requieren \emph{modificación del hardware}
\begin{itemize}
\item Noción de \emph{instrucciones privilegiadas}
\end{itemize}
\end{itemize}
\end{frame}

\begin{frame}[label={sec:orgcac3de4}]{Sistemas en lotes con \emph{spool}}
\begin{itemize}
\item ¿Spool? Bobina, o \emph{Simultaneous Peripherial Operations On-Line}
\item Cintas magnéticas
\begin{itemize}
\item Carga intermedia de tarjeta a cinta
\item Resultados a cinta para posterior impresión
\end{itemize}
\item Liberando los trabajos más lentos
\begin{itemize}
\item Empleo de equipos \emph{periféricos} especializados
\end{itemize}
\end{itemize}
\end{frame}

\begin{frame}[label={sec:org8ae1374}]{Sistemas multiprogramados}
\begin{itemize}
\item Diferentes etapas en la vida de un proceso: \emph{limitado por CPU},
\emph{limitado por entrada-salida}
\item Para maximizar el uso de recursos, ejecutar \emph{simultáneamente} varios
procesos
\begin{itemize}
\item Requiere cambios fuertes en el hardware
\item Protección de recursos — Espacio de memoria
\item Recursos \emph{estrictamente secuenciales} requieren \emph{bloqueos} para
ofrecer \emph{acceso exclusivo}
\end{itemize}
\item El monitor es invocado con mucha mayor frecuencia por los
temporizadores
\begin{itemize}
\item \emph{Cambios de contexto}
\end{itemize}
\item Interacción con el equipo: Se mantiene como el modelo en lotes
\end{itemize}
\end{frame}

\begin{frame}[label={sec:org0cfb4d1}]{Sistemas de tiempo compartido}
\begin{itemize}
\item 1960s: Sistemas \emph{interactivos} y \emph{multiusuarios}
\item Manejo de \emph{terminales} para la interacción (teletipos, CRTs)
\item Abstracciones de almacenamiento: Archivos, directorios en discos
\item Ventajas al programador:
\begin{itemize}
\item Interacción directa con el equipo
\item Edición interactiva
\item Compilación parcial
\item Ejecución inmediata
\item Bibliotecas de sistema
\end{itemize}
\item Complejidad técnica
\begin{itemize}
\item Requisito de múltiples cambios de contexto por segundo
\end{itemize}
\end{itemize}
\end{frame}

\begin{frame}[label={sec:orge2d4131},fragile]{Tipos de multitarea}
 \begin{description}
\item[{\emph{Cooperativa} o \emph{no apropiativa}}] (cooperative multitasking) Cada
proceso tiene control del CPU hasta que efectúa una \emph{llamada al
sistema} o indica al sistema que puede tomar el control (\texttt{yield})

\item[{\emph{Preventiva} o \emph{apropiativa}}] (preemptive multitasking) El reloj
del sistema interrumpe la ejecución de cada proceso transfiriendo
\emph{forzosamente} el control al sistema operativo
\end{description}
\end{frame}

\begin{frame}[label={sec:org5a0aaa7}]{Clases de procesos}
\begin{center}
¿Qué procesos son más importantes?
\end{center}
\begin{description}
\item[{Procesos interactivos}] El usuario tiene la \emph{experiencia} de la demora
\item[{Del sistema}] Procesos no postergables
\item[{Por nivel de usuario}] Hay usuarios más importantes que otros
\item[{\emph{…Asegún el sapo}}] Cada situación puede ameritar una política
diferente
\end{description}
\begin{center}
Los procesos se organizan en \emph{colas de prioridad} según las políticas
requeridas por cada sistema
\end{center}
\end{frame}

\section{Computadoras personales}
\label{sec:orgf23cc6d}
\begin{frame}[label={sec:org36d6798}]{Nacimiento de las computadoras personales}
\begin{itemize}
\item En los 1970s comienzan a aparecer las computadoras personales
\item En un principio, programadas a través de \emph{switches}, con resultados
a través de LEDs
\end{itemize}
\begin{figure}[htbp]
\centering
\includegraphics[width=0.5\textwidth]{../img/altair.jpg}
\caption{Microcomputadora Altair 8800 (1975, \(\approx\) US\$600)}
\end{figure}
\end{frame}

\begin{frame}[label={sec:org512f023}]{La era de los 8 bits (\(\approx 1977 - 1985\))}
\begin{columns}\begin{column}{0.6\textwidth}
\begin{itemize}
\item \emph{Microprocesadores} de 8 bits y miniaturización
\begin{itemize}
\item Salida a video (tipo TV)
\item Entrada por teclado
\item Entrada opcional por cinta, primeros \emph{diskettes} (discos
\emph{flexibles})
\end{itemize}
\item Programación en BASIC (intérprete en ROM)
\end{itemize}
\end{column} \begin{column}{0.4\textwidth}
\begin{figure}[htbp]
\centering
\includegraphics[width=\textwidth]{../img/commodore_pet.jpg}
\caption{Commodore Pet 2001 (1977)}
\end{figure}
\end{column}\end{columns}
\end{frame}

\begin{frame}[label={sec:org5295519}]{La era de los 8 bits (\(\approx 1977 - 1985\))}
\begin{itemize}
\item Comienzan a manejarse \emph{dispositivos}: Unidades de cinta, unidades de
disco, impresoras, modems, etc.
\item Muchas arquitecturas mutuamente incompatibles
\item Separan el \emph{entorno de desarrollo} del \emph{entorno de ejecución}
\begin{itemize}
\item Nace la \emph{distribución de binarios}
\end{itemize}
\item Explosión de la industria de los videojuegos
\end{itemize}
\end{frame}

\begin{frame}[label={sec:org2ca4b8d}]{La microcomputadora \emph{seria}: Familia PC (1981)}
\begin{columns}\begin{column}{0.3\textwidth}
\begin{figure}[htbp]
\centering
\includegraphics[width=\textwidth]{../img/ibmpc.jpg}
\caption{Computadora IBM PC modelo 5150 (1981)}
\end{figure}
\end{column}\begin{column}{0.6\textwidth}
\begin{itemize}
\item Primer computadora de una empresa \emph{seria}, orientada a su uso en
ambiente empresarial
\begin{itemize}
\item Sin color ni audio\ldots{} \emph{¿Para qué?}
\end{itemize}
\item Entorno primario de ejecución: \emph{Línea de comando} (PC-DOS, MS-DOS)
\item Al día de hoy sigue siendo la arquitectura predominante
\end{itemize}
\end{column}\end{columns}
\end{frame}

\begin{frame}[label={sec:orga1d57ec}]{Entorno gráfico (WIMP) (1984)}
\begin{columns}\begin{column}{0.3\textwidth}
\begin{figure}[htbp]
\centering
\includegraphics[width=\textwidth]{../img/mac128.png}
\caption{Apple Macintosh (1984)}
\end{figure}
\end{column}\begin{column}{0.6\textwidth}
\begin{itemize}
\item Ventanas, iconos, menúes, apuntador (\emph{Windows, Icons, Menus,
Pointer})
\item 1984: Apple Macintosh, primer sistema WIMP con éxito comercial
\item Multiprocesos, \emph{no multitarea}
\end{itemize}
\end{column}\end{columns}
\end{frame}

\begin{frame}[label={sec:org8688c89}]{Multitarea preventiva (1985)}
\begin{columns}\begin{column}{0.5\textwidth}
\begin{itemize}
\item Multitarea preventiva real: 1985 (Amiga, Atari ST)
\begin{itemize}
\item Sin hardware de protección de memoria
\end{itemize}
\item Los programadores \emph{tienen que considerar la concurrencia}
\end{itemize}
\end{column}\begin{column}{0.4\textwidth}
\begin{figure}[htbp]
\centering
\includegraphics[width=\textwidth]{../img/A500.jpg}
\caption{Commodore Amiga 500 (1987)}
\end{figure}
\end{column}\end{columns} \pause
\begin{figure}[htbp]
\centering
\includegraphics[height=2.5em]{../img/guru_meditation.png}
\caption{La \emph{meditación del gurú}}
\end{figure}
\end{frame}

\begin{frame}[label={sec:org98044a6}]{La profesionalización del escritorio}
\begin{itemize}
\item Fines de los 1980 — Intel 80486, Motorola 68040, PowerPC: Hardware
tan capaz como el de las \emph{estaciones de trabajo}
\item Reducción de las arquitecturas alternativas
\item Reemplazo paulatino de los sistemas operativos por otros más capaces
(o mejor \emph{mercadeados})
\begin{itemize}
\item Mención breve de casos: DOS, Windows y OS/2; AmigaOS y Atari ST;
NeXT y MacOS
\end{itemize}
\end{itemize}
\end{frame}

\begin{frame}[label={sec:orgc6f3e6e}]{Convergen Unix y las computadoras \emph{humildes}}
\begin{itemize}
\item Unixes históricos para computadoras personales: Xenix, A/UX, SCO.
\begin{itemize}
\item Muy limitados por su hardware
\item Precio desproporcionadamente alto; mejor ir por una \emph{estación de
trabajo}
\end{itemize}
\item Génesis del software libre \emph{ideológico}: GNU (1984)
\item 386/BSD: Primer sistema \emph{casi} libre
\item Linux (1991+); GNU/Linux
\item El mundo *BSD
\item \ldots{}Eventual muerte de las estaciones de trabajo
\begin{itemize}
\item Apollo, Digital, Sun, SGI, HP (PA/RISC), …
\end{itemize}
\end{itemize}
\end{frame}

\section{Dispositivos móviles}
\label{sec:orgd52164d}
\begin{frame}[label={sec:orgc15fd37}]{El mercado de los dispositivos (\emph{appliances})}
\begin{itemize}
\item 1990s: Agendas digitales inteligentes
\begin{itemize}
\item Que se van \emph{inflando} en computadoras completas
\end{itemize}
\item 2000s: Ruteadores, modems inteligentes, controladores de TV
(\emph{set-top boxes})
\item 2007+: Teléfonos celulares \emph{inteligentes}, tabletas — Por fin
exitosos (tras incontables fracasos)
\end{itemize}
\begin{center}
¿Cuál es la diferencia entre estos equipos y nuestras computadoras de
escritorio?
\end{center}
\end{frame}

\begin{frame}[label={sec:org2ce7d29}]{¿Qué define a los \emph{dispositivos móviles}?}
\begin{itemize}
\item Bajo consumo eléctrico
\begin{itemize}
\item Amplios \emph{estados de ahorro de energía}
\end{itemize}
\item Interfaz usuario limitada
\begin{itemize}
\item Mecanismos de entrada reducidos
\item Adecuación a dicha realidad en la interfaz usuario
\end{itemize}
\item \sout{Equipos limitados en rendimiento} \pause ¿En serio?
\end{itemize}
\end{frame}

\begin{frame}[label={sec:org39bef31}]{Primeros modelos}
\begin{columns} \begin{column}{0.7\textwidth}
Primer computadora \emph{portátil}: IBM 5100 (1975, 25Kg, pantalla de 5
pulgadas, US\$9,000)
\end{column} \begin{column}{0.3\textwidth}
\begin{center}
\includegraphics[width=.9\linewidth]{../img/IBM_5100.jpg}
\end{center}
{\tiny \href{https://commons.wikimedia.org/w/index.php?curid=16332671}{IBM 5100 (CC-BY Sandstein)}}
\end{column} \end{columns}

\begin{columns} \begin{column}{0.15\textwidth}
\begin{center}
\includegraphics[width=.9\linewidth]{../img/psion_organiser.jpg}
\end{center}
\end{column} \begin{column}{0.8\textwidth}
1984: \emph{Psion Organiser}: Computadora de bolsillo con reloj,
calculadora, base de datos, cartuchos de aplicaciones.  4KB RAM, 2KB
ROM, sin sistema operativo, programada en ensamblador. Antecedente
directo de \emph{Symbian}.
\end{column} \end{columns}
\end{frame}

\begin{frame}[label={sec:orged79a9b}]{1990s: El \emph{Asistente Digital Personal} (PDA)}
\begin{columns} \begin{column}{0.15\textwidth}
\begin{center}
\includegraphics[width=.9\linewidth]{../img/pda_sharp.jpg}
\end{center}
{\tiny \href{https://en.wikipedia.org/wiki/Sharp_Wizard#/media/File:SharpElectronicOrganiser-open.jpg}{Sharp Wizard}}
\end{column} \begin{column}{0.8\textwidth}
1990s: Popularización de los \emph{Asistentes Digitales Personales} (PDAs)
— Agendas de teléfonos, calendario, calculadora\ldots{} (No programables,
rara vez expandibles)
\end{column} \end{columns}

\begin{columns} \begin{column}{0.15\textwidth}
\begin{center}
\includegraphics[width=.9\linewidth]{../img/apple_newton.jpg}
\end{center}
{\tiny \href{https://en.wikipedia.org/wiki/Apple_Newton#/media/File:Apple_Newton-IMG_0454-cropped.jpg}{Apple Newton}}
\end{column} \begin{column}{0.6\textwidth}
Comenzó a desarrollarse la tecnología de entrada por pantalla, con
interfaces táctiles con \emph{stylus}, con grados de éxito\ldots{} Variables.

Aparece el soporte para \emph{ecosistemas} de software, comunicación entre
dispositivos, modos de \emph{hibernación}.
\end{column} \begin{column}{0.15\textwidth}
\begin{center}
\includegraphics[width=.9\linewidth]{../img/palm_pilot.png}
\end{center}
{\tiny \href{https://en.wikipedia.org/wiki/Palm_(PDA)}{Palm Pilot}}
\end{column} \end{columns}
\end{frame}

\begin{frame}[label={sec:orgdfab061}]{1995 – 2010: Puliendo la telefonía}
\begin{columns} \begin{column}{0.15\textwidth}
\begin{center}
\includegraphics[width=.9\linewidth]{../img/kyocera_6035.jpg}
\end{center}
{\tiny \href{https://en.wikipedia.org/wiki/Kyocera_6035#/media/File:Kyocera6035.jpg}{Kyocera 6035}}
\begin{center}
\includegraphics[width=.9\linewidth]{../img/n810.jpg}
\end{center}
{\tiny \href{https://en.wikipedia.org/wiki/Nokia_N810#/media/File:N810-open.jpg}{Nokia N810}}
\end{column} \begin{column}{0.6\textwidth}
\begin{itemize}
\item Aparición de los \emph{teléfonos inteligentes}
\item Personalidad dividida, dos computadoras en consonancia (señalización
celular e interfaz usuario)
\item Tendencia hacia interfaces \emph{multitouch}
\item Enfoque en eficiencia de consumo eléctrico
\end{itemize}
\end{column} \begin{column}{0.15\textwidth}
\begin{center}
\includegraphics[width=.9\linewidth]{../img/treo.jpg}
\end{center}
{\tiny \href{https://es.wikipedia.org/wiki/Treo#/media/File:Treo680sp.jpg}{Palm Treo}}
\begin{center}
\includegraphics[width=.9\linewidth]{../img/iphone.jpg}
\end{center}
{\tiny \href{https://en.wikipedia.org/wiki/IPhone_3G}{iPhone}}
\end{column} \end{columns}
\end{frame}

\begin{frame}[label={sec:org5d1cfe8}]{2010– Maduración del mercado móvil}
\begin{itemize}
\item Consolidación de sistemas operativos disponibles
\begin{itemize}
\item Dominantes: Android (\(\leftarrow\) Linux), iOS (\(\leftarrow\) MacOS)
\item Desaparecen: Symbian, Windows Phone, Firefox OS
\end{itemize}
\item Universalización de la interfaz
\begin{itemize}
\item \ldots{}Levante la mano quien \emph{no} tiene uno consigo\ldots{}
\end{itemize}
\end{itemize}
\end{frame}

\begin{frame}[label={sec:org4a4ed8b}]{Características generales del segmento}
\begin{itemize}
\item Almacenamiento en estado sólido (vs. discos magnéticos giratorios)
\item Interfaz usuario: Multitarea, pero \emph{monocontexto}
\item Consumo eléctrico
\begin{itemize}
\item Menor consumo
\item Más \emph{estados de descanso}, detección de patrones de uso acorde
\item Salto constante entre estados
\end{itemize}
\item No asumen estabilidad: Adecuación a un entorno cambiante
\item Disponibilidad de aplicaciones: \emph{Jardín amurallado}
\end{itemize}
\end{frame}

\begin{frame}[label={sec:org08c4e2a}]{Arquitecturas hardware en boga hoy}
\begin{itemize}
\item En escritorios y servidores: Derivada de PC (Intel x86)
\item En dispositivos embebidos ARM (ocasionalmente MIPS; ojo ⇒ OpenRISC)
\begin{itemize}
\item ARM como alternativa de bajo consumo para servidores
\end{itemize}
\item Sigue habiendo un importante espacio a controladores que no
requieren sistema operativo (p.ej. Arduino, ARM perfiles R/M,
ASICs\ldots{})
\begin{itemize}
\item Probablemente siempre lo habrá
\end{itemize}
\end{itemize}
\end{frame}
\end{document}